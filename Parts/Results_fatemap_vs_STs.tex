
I found a few cases in \textit{Ciona} in which cells with the same fate where contained in different STs. As explained in section X, it would be expected that a fate map would largely coincide with a gene expression map. This analysis could not be made in \textit{Drosophila} as the gene expression data is not a the single level resolution.

A lack of correspondence, as it was found in here, could be due to: 1) cells whose fate is disproportionally affected or determined by a small number of genes (as this analysis reflect quantitative differences at the level of hundreds of expressed genes but can not distinguish between the relative importance of each gene) or 2) cells that although having a restricted fate at a certain stage their differentiation is not complete (at the level of gene expression).
%
An example of the latter is a ST in the 112-cell stage (in magenta; \ref{fig:Art-II-territories} B; II, Fig. S8) that contains precursors of the notochord (A8.5, A8.6, A8.13, and A8.14, B8.6) and mesenchyme (B8.5) \citep{Tokuoka2004}.
The latter come from a secondary notochord/mesenchyme bipotential cell (B7.3). It has been reported that the expression of Twist-like 1, necessary for mesenchyme differentiation, starts at this stage \citep{Imai2003}.
This evidence, together with the inclusion of the mesenchyme cell in this otherwise exclusively notochord territory (primary and secondary), seems to indicate that the differentiation of cell pair B8.5 as mesenchyme is still incomplete at this stage.

\subsubsection{Gene expression dynamics in cell-lineages}

During \textit{Ciona} early embryogenesis, I analysed the gene expression similarity between lineage-related cells i.e., between daughters cells and between mother/descendants cells (II, Fig. 8).
In general, cells are more closely genetically to their sister cells than to their mother/descendants, which is also reflected in the clustering of STs by stages discussed before.
I also found that at the 64-cell stage, cells that show more genes expressed differently than their ancestors are neural fated cells. This could be related with the change from unrestricted state of these cells at this stage (i.e., their descendants will give rise to different cell fates) to a restricted state in the next stage (112-cell stage) \citep{Imai2006}.
Therefore, it could be hypothesized that when a cell changes from a unrestricted to a restricted cell fate state, a major change in gene expression should be evident when following gene expression dynamics of its cell-lineage.  