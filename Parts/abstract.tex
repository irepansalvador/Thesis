
Embryonic development has amazed scientists and philosophers for many centuries
%
Many reasons have been evoked for the apparent increase of the complexity (even when complexity has eluded an unique definition).
Aristotle claimed a "vital force" was the guiding force driving embryogenesis.
%
Vitalism was eventually substituted by preformationism or epigenesis explanations.

Since Charles Darwin, many authors have considered the embryo as an important source of adaptations
%
Some regarded each embryonic stage to have adaptation as the result of the constant interaction of the embryo with its environment (whether external or internal).
%
Others have distinguished the adaptations in the embryo from those from the adult, stating that the former are product of internal selection, instead of being product of natural selection, as the latter.

(In relation to this) since the 19th century certain patterns have been recognized when comparing the embryonic development (or trajectories) of different species. 
%
According to the von Baer pattern (or von Baer law), when comparing animals of different phyla, early development is very similar and  

From the 1980's developmental genes have been used to shed light in evolutionary processes.
%
Since then, information in many specific genes or gene networks have revealed important information in selected model species.

I will use gene expression information under a statistical approach, to estimate both complexity and adaptation in the embryo.
This means no special emphasis on specific genes or pathways is made. 

Instead, my focus is to gain a broad picture on the spatiotemporal patterns of gene expression complexity and adaptation.

Hopefully, this can serve as a proxy of these processes at the phenotypic level.

In brief, the results presented here...