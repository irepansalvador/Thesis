
Embryonic development has amazed scientists for centuries.
%
Many reasons have been suggested for the perceivable increase in complexity in development, during which a single cell into a larva or an adult.
% (even when complexity has eluded an unique definition).
%Aristotle claimed a "vital force" guided embryogenesis. %Eventually, 
%Vitalism was substituted by preformationism and epigenesis explanations in the 18th century.
%
%Since the 1980's, when it became clear that genes play a key role in embryogenesis, the causal mechanisms for such complexity increase are being searched for at the cellular and molecular level.
%
%Using the number of cell types as complexity measure, its increase during development is self-evident: in mammals, the embryo begins with one cell type and concludes with up to 200 cell types. 
%This process can also be called embryo compartmentalization.
%From many discoveries since the 1980's, it has become clear that some genes play a key role in the embryo compartmentalization. 
%At the level of gene expression, it its assumed that the expression of the genes, that in early development are expressed in large domains on the embryo,
At the level of gene expression, 
it its assumed that genes change from being expressed in large spatial domains of the embryo in early development to spatially restricted domains (e.g., tissues, cells) in late development.
%in early development,  Then, at later stages, their expression 
%, until some genes are expressed in a cell-specific manner.
%
For many developmental genes, the spatio-temporal expression dynamics
%, and how it relates to the embryo compartmentalization, 
has been thoroughly described.
It is not clear however, if the global dynamics are similar,
% deployed in development, 
or if there are differences between types of genes or between species.

%-----------------------------
\hfill\break 
%Adaptations are among the causes that have been proposed for the increase in complexity.
%To some authors, adaptive reasons could be the cause for such increase in complexity.
Adaptive reasons have been also said to be the cause for the increase in complexity.
%
%A change during development would be adaptive if it increases the fitness of its bearer, even when the effect of this developmental change is realized until the adult or larva.
%The increase in complexity (or any change in development) has been considered for some authors as the adaptations of the embryo to its environment.
%
%
Adaptations could be estimated with molecular evolution methods based on the analysis of genes expressed in different developmental stages or regions in the embryo. 
%at the DNA sequence level with 
%molecular evolution methods. 
These methods estimate adaptive changes at the DNA sequence level assuming that
%These methods are based on the quasi-neutral evolution model that assumes that
%Under this model, an adaptive change that has been caused by a genetic mutation, can be traced in the DNA sequence, as 
a positive selected site would show less variance than other sites evolving neutrally.
%
%Recently, methods of population genomics use divergence ans polymorphism (inter-specific and intra-specific variation) to measure more precisely the proportion of adaptive mutation at the molecular level.
%
%----------------------------
Different developmental stages might show distinct levels of positive or stabilizing selection, 
%leading to some specific patterns might be recognizable
that could be related to inter-specific divergence patterns proposed by the von Baer's laws or the hourglass model. 
%, when comparing the development of different species in a group.
% specific patterns might appear (patterns of what?) when comparing the divergence between different species.
%Two related models currently under much discussion? are:
%This is related to already proposed models of divergence: 
%1)von Baer's laws and the hourglass model.
The former states that the development of two species of a phylogenetic group would be very similar in early stages and increasingly divergent in subsequent stages. In contrast, the latter states that development is less divergent (more conserved) at mid development.

%Such patterns could be due to selection or constrant..
%The print in natural selection on the genome is useful to discern between these scenarios.
%Some regarded each embryonic stage to have adaptation as the result of the constant interaction of the embryo with its environment (whether external or internal).
%Others have distinguished the adaptations in the embryo from those from the adult, stating that the former are product of internal selection, instead of being product of natural selection, as the latter.
%------------------------------------------
\hfill\break 
In here, I analysed gene expression information to estimate both complexity and adaptation in the embryo using a statistical approach.
%, i.e., no special emphasis on specific genes or pathways is made. 
%Instead, my intention is to provide a broad picture on the spatiotemporal change in complexity and adaptation in the embryo.
%
%I take advantage of available databases of gene expression.
%I analyze complexity using two popular developmental biology models: Drosophila melanogaster and Ciona intestinalis.
To measure complexity, I developed quantitative measures of spatial complexity and used them in publicly available gene expression data (thousands of in situ hybridization experiments) % of two popular developmental biology models: 
 in \textit{Drosophila melanogaster} and \textit{Ciona intestinalis} from the BDGP/FlyExpress and ANISEED databases respectively.
%
To estimate adaptation, I combined diverse \textit{D. melanogaster} gene expression data (modENCODE,  in situs from the BDGP/FlyExpress and gene expression data based on a controlled vocabulary of the embryo anatomy) with population genomic data (from the DGRP project). Using the DFE-alpha method 
(which uses coding-region polymorphism and divergence to estimate the proportion of adaptive changes)
, I charted a spatial map on adaptation of the fruit fly embryo's anatomy. 
Finally, I analysed the pattern of positive selection through the entire life cycle of \textit{D. melanogaster} and how it correlated with specific genomic determinants (e.g., gene structure, codon bias)

\hfill\break 
Briefly, I found that \textit{Drosophila} and \textit{Ciona} complexity increases non-linearly with the major change in complexity being before and after gastrulation, respectively. 
In both species, transcription factors and signalling molecules showed an earlier compartmentalization, consistent with their proposed leading role in pattern formation.
%
In \textit{Drosophila}, gonads and head showed high adaptation during embryogenesis, although pupa and adult male stages exhibit the highest levels of adaptive change, and mid and late embryonic stages show high conservation, showing an HG pattern.
%Furthermore, I propose gene structure complexity as an explanation for this HG-like pattern.
Furthermore, I propose that the explanation for the lack of conservation in the early stages could be a relaxation of natural selection, and that the hourglass pattern could be explained by gene structure complexity.