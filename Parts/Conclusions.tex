

The study of organismal complexity during embryonic development presented here shows that there are commonalities and differences between D. melanogaster and C. intestinalis. Both species showed a non-linear increase in all complexity measures, while the most remarkable difference is the timing of the major change in complexity, which is earlier in D. melanogaster (around gastrulation).
Another common pattern is the early increase in complexity when considering only transcription factors or growth factors (or other signalling molecules). This confirms the special role these genes have in early metazoan development. It could be therefore expected that the evidence presented here, regarding these type of genes, should be observed also in other species. 

One important result of this work is that within each species, the three complexity measures showed a similar pattern (even when it would not be necessarily the case; see section X). This means that altogether, these measures (compartmentalization, disparity and roughness) are reflecting a global pattern of increase in complexity in each species. Therefore, it could be hypothesized that a similar increase in complexity would be found using alternative measures of complexity (e.g., spatial entropy). Further analysis would be required to test this hypothesis.

In here, I have taken advantage of the great amount of information about developmental gene expression that has accumulated in many years from collective efforts of the developmental biology community. The great amount of accumulated data allows to shift the focus to a systemic approach in which the global statistical properties of development can be investigated.


---


The elaboration of an adaptation map on the fruit fly embryo as presented here can be considered a proof of concept of how the combination of scientific fields like evolutionary developmental biology and population genomics, and new techniques such as the phylostratigraphy, can be useful to give a fresh view on an old problem.
Using these maps, it was possible to visually identify that the center (inner part) of the embryo expresses a more conserved and older transcriptome, while the outside of the embryo expresses more phylogeneticaly young and less conserved genes. This evidence seems to coincide with the hypothesis of the antecedence of the endoderm with respect to the ectoderm (REF Hashimshony). 
-late development -> larva

The estimation of adaptation over the entire life cycle of Drosophila, a presented here, supports the HG model of development. This analysis is different from previous ones in the use of a more complete spatio-temporal dataset (based on modencode) and a method that uses inter and intra-specific DNA coding variation to estimate, with an unprecedented precision, the proportion of adaptive changes.

Furthermore, as a result of this work is hypothesized that the hourglass model can be partially explained by various genomic features. More work is necessary to test this hypothesis.


Most of the results derived from this study are in agreement (or are consistent) with the current knowledge and expectations from the development of these model organisms.. but...

---
OPEN QUESTIONS

The emergence of new techniques, like "spatial transcriptomics" of tissue sections at single-cell resolution could make possible to have information, derived from a single experiment, of all the genes expressed in a 2D section of an embryo. The application of the measures presented here could be apply to this kind of data in a straightforward manner, sikving in this way the limitations in resolution of the work presented here.

It is important to mention that this work has used differential gene expression in the embryo and its spatial distribution as a tool to investigate complexity. However, embryonic development can not be reduced to differential gene expression. Cellular behaviours and the physical properties of the cells and tissues have also a causal role in the developmental process. It would be interesting to be able to measure the different apportionment to complexity increase of the different developmental mechanisms.
%%%%%% for results of compartmentalization
%
%It could be expected that these early increase in complexity in drosophila is shared by all insects with a syncitial blastoderm stage. It could be that there are differences between these based on the number of cell divisions until the blastoderm is cellularized. It is known that Drosophila cellularizes relatively late (so there is more time for patterning within the syncitial bastoderm). In contrast, the desert locust (Schistocerca gregaria) cellularization occurs very early, before the formation of the blastoderm (REF Ho).
%
%The rapid development in drosophila could be due to selective pressures on the time of development, caused by the the eggs being layed ephemeral resources such as decaying fruits (this classical view has been chalenged recently; reviewed in REF Prasad). 
%The embryonic development of D. melanogaster takes around 1 day at 22 degrees. Even when developmental time can vary in a three fold manner depending on the Drosophila species and temperature (D. virilis embryonic development at 18 C takes more than 45 hours, compared to less tha 15 hours at 30 C in D. ananassae: REF Kuntz), it can be considered fast compared the dessert locust. In the locust, embryonic development takes around two weeks in calid environments in West Africa but it can take up to 70 days in the cooler temperatures of North Africa (REF locushanbbook).
%
%Also, it would be interesting to know if there are some differences between the two main modes of segmentation in insects, i.e., short germband and long germband. The red bettle (Tribolium castaneum) is most popular short germband inset that serves as a developmental biology model organism. Many valuable resources have become available in the last decades/years (REF Bucher). 
%
%
%The period of egg development, between laying and hatching, is called the incubation period. The rate at which eggs develop varies according to the soil temperature. For example, in the summer breeding areas of West Africa, the Red Sea coast and lowland India the incubation period takes 10-14 days but this is extended to 25-30 days in the cooler spring breeding areas of central Arabia, southern Iran and Pakistan while in North Africa it can take as long as 70 days in exceptionally cold weather
