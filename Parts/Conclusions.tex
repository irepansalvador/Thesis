
The study of organismal complexity during embryonic development presented here shows that there are commonalities and differences between \textit{D. melanogaster} and \textit{C. intestinalis}. Both species showed a non-linear increase in all complexity measures, while the most remarkable difference is the timing of the major change in complexity, which is earlier in \textit{D. melanogaster} (around gastrulation).
Another common pattern is the early increase in complexity when considering only transcription factors or growth factors (or other signalling molecules). This confirms the special role these genes have in early metazoan development. It could be therefore expected that, based on the evidence presented here, the same pattern when considering these type of genes should also be observed in other species. 

One important result of this work is that within each species, the three complexity measures showed a similar pattern (even when it would not be necessarily the case; see section \ref{measures_relations}). This means that altogether, these measures (compartmentalization, disparity and roughness) are reflecting a global pattern of increase in complexity in each species. Therefore, it could be hypothesized that a similar increase in complexity would be found using alternative measures of complexity (e.g., spatial entropy). Further analysis would be required to test this hypothesis.
Also, the Synexpression Territories analysis allowed to "reconstruct" the main embryonic differentiation events in both species in a consistent manner with the current knowledge of the development of these model organisms and without focusing in specific genes.

The elaboration of an adaptation map on the fruit fly embryo can be considered a proof of concept of how the combination of diverse fields like evolutionary developmental biology and population genomics, and new techniques such as the phylostratigraphy, can be useful to give a fresh view on an old problem.
Using these maps, it was possible to visually identify that the center (internal part) of the embryo expresses a more conserved and older transcriptome, while the outside (external part) expresses phylogeneticaly younger and less conserved genes. This evidence seems to support the hypothesis of the antecedence of the endoderm with respect to the ectoderm \citep{Hashimshony2014}. It would be interesting to extend this adaptation mapping analysis for the entire development (until the adult stage) as it could be that in later stages, different structures or organs have been under positive or negative selection.

The estimation of adaptation over the entire life cycle of \textit{D. melanogaster}, as presented here, supports the HG model of development. We find, as other analyses previously have, that the mid-embryogenesis is highly conserved. The work presented here is different from previous ones in that it uses a more complete spatio-temporal dataset and a method that uses inter and intra-specific DNA coding variation to estimate, with an unprecedented precision, the proportion of adaptive changes.
Furthermore, as a result of this work is hypothesized that the hourglass model can be best predicted by various genomic features. However, further work is necessary to test this hypothesis.

The observed patterns of complexity and adaptation/conservation throughout the embryonic development of \textit{D. melanogaster} might be intricately connected.
%
%which reflects that the different parts of the embryo express progressively more different combinations of genes, 
%is likely related with the expression, at this specific developmental period, of genes with multiple expression domains.
The increase in spatial disparity of gene expression in late embryogenesis (Fig. \ref{fig:Art-I-3measures}) likely reflects the expression of genes with multiple spatial domains.
%
Genes expressed in many different places and times during development likely require an elaborate genetic structure (reflected in their exons number, intron length, transcripts number) that could permit such complex spatio-temporal expression regulation.
%
It could be then hypothesized that a mutation in such a gene would have high pleiotropic effects (as it would affect many different parts of the embryo) which could result in stabilizing selection against mutational variation \citep{Raff1996,Galis2002}.
The correlation between specific genetic features intuitively related to spatio-temporal regulation of gene expression and the high level of conservation at late embryogenesis seems to support this hypothesis. 
%A more precise analysis, which could disentangle all these variables is however required.

%
%SERIA BUENO TAMBIEN HACER ALGUN TIPO DE CONEXION ENTRE LO DE LA COMPLEJIDAD Y LO DE LA ADAPTACION EN EL ESPACIO O EN EL TIEMPO (EN EL TIEMPO SERIA MAS FACIL Y RELACIONARLO CON LO QUE DECIMOS DE LA CONSTRICCION EN LOS GENES DEL DESARROLLO DADO LA COMPLEJIDAD DE SU REGULACION.

%This thesis confirms that a statistical approach in developmental biology can provide valuable information on fundamental processes by describing their properties at a statistical level, beyond the role of individual genes.
In here, analysing publicly available databases, I have quantified how complexity and compartmentalization increase during development of two species (\textit{D. melanogaster} and \textit{C. intestinalis}) and estimated the rate of adaptive evolution over the entire embryo's anatomy and in the whole life cycle in \textit{D. melanogaster}.
Thus, the work presented here confirms that a statistical approach in developmental biology can provide valuable information on fundamental processes by describing their properties at a statistical level, and therefore allows to attain a global view that transcends the role of individual genes.

%AQUI TAMBIEN PODRIAS DECIR QUE TU TESIS CONFIRMA QUE LA STATISTICAL DEVELOPMENTAL BIOLOGY PUEDE DECIR COSAS INTERESSANTES MAS ALLA DEL PAPEL DE CADA GEN (RELACIONADO CON LO QUE DICES AL PRINCIPIO DE LA INTRO SOBRE ESTO.
%In here, I have taken advantage of the great amount of information about developmental gene expression that has accumulated in many years from collective efforts of the developmental biology community. This great amount of accumulated data allows to shift the focus from single genes to a systemic approach in which the global statistical properties of development can be investigated.


\subsection*{Future directions}

The approach presented here could be applied to other model organisms for which gene expression databases, similar to the databases I analysed in here, are available (e.g., \textit{mouse}, \textit{Xenopus}). Also, it could be applied to model systems in developmental biology for which there is sufficient spatio-temporal gene expression data available, like the \textit{Drosophila} wing imaginal disc or the vertebrate limb bud.

The emergence of new techniques, like "spatial transcriptomics" of tissue sections at single-cell resolution \citep{Stahl2016} could make possible to have information, derived from a single experiment, of all the genes expressed in a 2D section of an embryo. The application of the measures presented here could be applied to data derived from this new technique in a straightforward manner, solving the limitation in resolution of the work presented here.

It is important to mention that this work has used differential gene expression in the embryo and its spatial distribution as a tool to investigate complexity. However, embryonic development can not be reduced to differential gene expression. Cellular behaviours and the physical properties of the cells and tissues have also a causal role in the developmental process. It would be interesting to be able to measure the differential apportionment to complexity increase of the different developmental mechanisms.

The increase of organismal complexity and the study of adaptation during development remain fascinating topics after many centuries, and still offer many open questions to be solved. The incredibly fast pace of data generation, the development of new techniques and sophisticated methods give hope to finally open the black box of development.

%%%%%% for results of compartmentalization
%
%It could be expected that these early increase in complexity in drosophila is shared by all insects with a syncitial blastoderm stage. It could be that there are differences between these based on the number of cell divisions until the blastoderm is cellularized. It is known that Drosophila cellularizes relatively late (so there is more time for patterning within the syncitial bastoderm). In contrast, the desert locust (Schistocerca gregaria) cellularization occurs very early, before the formation of the blastoderm (REF Ho).
%
%The rapid development in drosophila could be due to selective pressures on the time of development, caused by the the eggs being layed ephemeral resources such as decaying fruits (this classical view has been chalenged recently; reviewed in REF Prasad). 
%The embryonic development of D. melanogaster takes around 1 day at 22 degrees. Even when developmental time can vary in a three fold manner depending on the Drosophila species and temperature (D. virilis embryonic development at 18 C takes more than 45 hours, compared to less tha 15 hours at 30 C in D. ananassae: REF Kuntz), it can be considered fast compared the dessert locust. In the locust, embryonic development takes around two weeks in calid environments in West Africa but it can take up to 70 days in the cooler temperatures of North Africa (REF locushanbbook).
%
%Also, it would be interesting to know if there are some differences between the two main modes of segmentation in insects, i.e., short germband and long germband. The red bettle (Tribolium castaneum) is most popular short germband inset that serves as a developmental biology model organism. Many valuable resources have become available in the last decades/years (REF Bucher). 
%
%
%The period of egg development, between laying and hatching, is called the incubation period. The rate at which eggs develop varies according to the soil temperature. For example, in the summer breeding areas of West Africa, the Red Sea coast and lowland India the incubation period takes 10-14 days but this is extended to 25-30 days in the cooler spring breeding areas of central Arabia, southern Iran and Pakistan while in North Africa it can take as long as 70 days in exceptionally cold weather
