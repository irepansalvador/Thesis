
During the last decades the scientific community has witnessed the flourishing of modern developmental biology (although developmental biology can not be considered a young scientific discipline, as its roots come from centuries ago from embryology and anatomy). Since the 1980's crucial discoveries \citep{Gilbert1998} have improved our understanding of the developmental process in many model organisms.

Most of the modern developmental biology studies use an "individualistic" approach \citep{Davidson2009}, e.g., focusing only on the description of some gene's effect on the development of a specific structure or the role of a gene in a specific signalling pathway.
%
This individualistic approach has increased substantially the knowledge in the developmental biology field and has accumulated a great amount of gene expression information in many years of collective efforts of the developmental biology community.
The emergence of methods like DNA microarrays extended the determination of the expression of a single gene to a genomic level, allowing new systemic approaches to study gene expression during development. An example of the results obtained by these approaches is the identification of groups of temporal co-expressing genes during development (e.g., \citealp{Arbeitman2002,Hooper2007}).

The majority of the systemic approaches on gene expression during development have focused on the temporal analysis of expression, without considering the spatial distribution of the expressing genes in the embryo (there are however some noteworthy studies that have analysed the spatial patterns of gene expression during development, e.g., \citealp{Gurunathan2004,Tomancak2007,Frise2010,Crombach2012,Konikoff2012}).
% Also, high-throughput analysis of mRNA hybridization have allowed the classification of developmental genes based on their tissue-specificity \citep{Tomancak2007}. 
%\textbf{CITAR LO DE TEMPORAL Y Q NO HAY ESPACIAL. FRISE ONLY FATE.}

The analysis of the spatial patterns of gene expression is now facilitated by recent high-throughput in situ hybridization approaches \citep{Tomancak2002,Pollet2003,Imai2004,Christiansen2006,Lecuyer2007,Tassy2010}, which have not only further increased the amount of spatio-temporal gene expression data during development of some model organisms, but also allow straightforward comparisons between gene expression patterns using computational methods.
Therefore, the availability of gene expression at a genomic level allows to shift the focus of developmental biology from the study of single genes to a systemic approach in which the global statistical properties of development can be investigated.

The individualistic approach is also common in studies that aim to detect natural selection. Most studies that directly search for adaptation at the phenotypic level analyse only a single trait or a small number of traits \citep{Hoekstra2001,Hereford2004}. However, there is no study that has estimated natural selection over the entire body of an organism.
As any adaptive change in the phenotype is expected to be partially caused by genetic mutation, an alternative to detect natural selection is the analysis of DNA sequences of genes expressing differentially in different parts of an organism's body. This could be extended to different stages in the life cycle of an organism if there is enough spatio-temporal gene expression information.



%For example, in the early 80's, as techniques to identify whether a gene is expressed in a specific stage of development or tissue became available (e.g., \citet{Bialojan1984}), lead to an many studies showing the expression of one gene (or genes from a gene family) during development. The characterization of the spatial or temporal expression of a gene during development allowed the determination of the link between gene products and their molecular function \textbf{ESTO DEL LINK MOLECULAR O CAMBIARLO O MODIFICARLO}.

%
%In this work, I have analysed publicly accessible spatio-temporal gene expression data of two model organisms, \textit{Drosophila melanogaster} and \textit{Ciona intestinalis}, together with population genomics data of \textit{D. melanogaster}.
%Using a statistical approach, I address three questions:
%%%%%%%%%%%%%%%%%%%%%%%%%%%%%%%%%%%%%%%%%%%%%%%%%%%%%%%%%%%%%%%%%%%%%%%%%%%
%\begin{enumerate}
%\item How do complexity and compartmentalization increase in the embryo during development?
%\item Can adaptation be found in specific anatomical parts of the embryo or developmental stages? 
%\item Is the Hourglass model (which states that there is less amount of inter-specific variation in mid-development; see section \ref{hourglass}) supported by evidence of natural selection at the DNA sequence level?
%\end{enumerate}
%%%%%%%%%%%%%%%%%%%%%%%%%%%%%%%%%%%%%%%%%%%%%%%%%%%%%%%%%%%%%%%%%%%%%%%%%%%
%
%These questions have been selected for the great interest they have aroused in the scientific community since the early days of developmental and evolutionary biology.
%%
%The work presented here is based on and uses concepts from three main biology fields: 
%developmental biology, evolutionary biology and population genetics.
%Nowadays, the combination of these scientific fields constitute multiple research programmes. Modern evolutionary developmental biology (evo-devo) is the explicit combination of the first two fields.
%	\nomenclature{evo-devo}{Evolutionary developmental biology}

%However, these fields have not always gone hand in hand. Some decades ago, there was a clear conceptual and epistemological separation between evolutionary biology (mostly practised by geneticists) and developmental biology, even though embryology (which slowly transformed into developmental biology in the middle of the 20th century, see \citealp{Horder2010}) was considered crucial for the study of evolution in the 19th century.
%
%In the following section, I will give a brief introduction of the scientific and philosophical origins of developmental biology, with special attention to its relations with evolutionary biology and genetics (for a comprehensive review on this issue, see: \citealp{amundson2005changing}; \citealp{gilbert1991conceptual}), and to some of the concepts I will use in this dissertation.
In the next subsections, I will make an introduction of the study of complexity and adaptation during embryonic development, emphasizing the methods and concepts that have been previously (or could be potentially) used to analyse both. 
%
Before I do this, it might be useful to define what is development. So firstly, I will address this apparently simple question.