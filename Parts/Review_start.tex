
During the last decades the scientifc community has witnessed the flourishing of developmental biology. Since the 1980's crucial discoveries \citep{Gilbert1998} have not only improved our understanding of the developmental process, but also changed the perspective of the explanatory role of development in biology.

Most of the modern developmental biology studies use an "individualistic" approach \citep{Davidson2009}, e.g., focusing only on the description of some gene's effect on the development of a specific structure or the role of a gene in a specific signalling pathway.
%
This "individualistic" approach, although having increased substantially the knowledge in the developmental biology field, is often not suited to gain a global view that could serve to answer interesting general questions in Biology. 
%
In order to attain this global view a systemic statistical approach would be fruitful.
%
%There have been some previous systemic approaches in developmental biology that have proved their usefulness.
%%Davidson, among others, started the analyses of developmental transcriptional systems in the 70s (REF).
%Nusslein Vollhard (REF) recent (Bucher ref)

In this work, I have analysed publicly accessible spatio-temporal gene expression data of two model organisms, \textit{Drosophila melanogaster} and \textit{Ciona intestinalis}, together with population genomics data of \textit{D. melanogaster}.
Using a systemic statistical approach, I address three questions:
%%%%%%%%%%%%%%%%%%%%%%%%%%%%%%%%%%%%%%%%%%%%%%%%%%%%%%%%%%%%%%%%%%%%%%%%%%
\begin{enumerate}
\item How do complexity and compartmentalization increase in the embryo during development?
\item Can adaptation be found in specific anatomical parts of the embryo or developmental stages? 
\item Is the Hourglass model (which states that there is less amount of inter-specific variation in mid-development; see section \ref{hourglass}) supported by population genomics data?
\end{enumerate}
%%%%%%%%%%%%%%%%%%%%%%%%%%%%%%%%%%%%%%%%%%%%%%%%%%%%%%%%%%%%%%%%%%%%%%%%%%

These questions have been selected for the great interest they have aroused in the scientific community since the early days of developmental and evolutionary biology.
%
The work presented here is based on and uses concepts from three main biology fields: 
developmental biology, evolutionary biology and genetics.
Nowadays, the combination of these scientific fields form multiple research programmes. Indeed, modern evolutionary developmental biology (evo-devo) is the explicit combination of the first two fields.
	\nomenclature{evo-devo}{Evolutionary developmental biology}

However, these fields have not always gone hand in hand. Some decades ago, there was a clear conceptual and epistemological separation between evolutionary biology (mostly practised by geneticists) and developmental biology, even though embryology (which slowly transformed into developmental biology in the middle of the 20th century, see \citealp{Horder2010}) was considered crucial for the study of evolution in the 19th century.

In the following section, I will give a brief introduction of the scientific and philosophical origins of developmental biology, with special attention to its relations with evolutionary biology and genetics (for a comprehensive review on this issue, see: \citealp{amundson2005changing}; \citealp{gilbert1991conceptual}), and to some of the concepts I will use in this dissertation.

Before that, it might be useful to define what development is, so first, I will address this apparently simple question.