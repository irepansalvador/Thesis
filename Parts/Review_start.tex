\setlength{\epigraphrule}{0\p@}
\setlength{\epigraphwidth}{.7\textwidth}
\epigraph{\textit{" It is not enough to see that horse pulling a cart past
the window as the good working horse it is today; the picture
must also include the minute fertilised egg, the embryo in its
mother's womb, and the broken-down old nag it will eventually
become."}}{C. H. Waddington 1957}

This work is based and uses concepts from three main biology fields: 
developmental biology, evolutionary biology and genetics.
Nowadays, the combination of these scientific fields form multiple research programmes. 
In evolutionary developmental biology (evo-devo), the explicit combination of the first two fields, has been recognized \citep{Muller2007}.
	\nomenclature{evo-devo}{Evolutionary developmental biology}
However, these fields have not always gotten along well. 
Some decades ago, there was a clear conceptual and epistemological separation between evolutionary biologists (mostly geneticists) and developmental biology, even when embryology (which slowly transformed into developmental biology in the middle of the 20th century, see \citealp{Horder2010}) was considered crucial for the study of evolution in the 19th century.

In the following section, I will make a brief introduction of the scientific and philosophical origins of developmental biology, with special attention to its relations with evolutionary biology and genetics (for a comprehensive review on this issue, see: \citealp{amundson2005changing}; \citealp{gilbert1991conceptual}).
Before that, it might be useful to define what development is, so firstly, I will address this apparently simple question.