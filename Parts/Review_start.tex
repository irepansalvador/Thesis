This work is based and uses concepts from three main biology fields: 
developmental biology, evolutionary biology and genetics.
Nowadays, the union of these scientific fields form multiple research programmes. 
Only in evolutionary developmental biology (evo-devo), the explicit union of the first two fields, at least four major research programmes have been recognized (Muller, 2006).
However, these fields have not always gotten along well. 
Some decades ago, there was a clear conceptual and epistemological separation between evolutionary biology and developmental biology, even when embryology (which slowly transformed into developmental biology in the middle of the 20th century, see \citealp{Horder2010}) was considered crucial for the study of evolution in the 19th century.

In the following section, I will make a brief introduction of the scientific and philosophical origins of developmental biology, with special attention to its relations with evolutionary biology and genetics (for a comprehensive review on this issue, see: \citealp{amundson2005changing}; \citealp{gilbert1991conceptual}).
Before that, it might be useful to define what development is, so firstly, I will address this apparently simple question.