%objective
With a time series cluster analysis \citep{Ernst2006} of the relative area of expression, I found the eight main spatio-temporal profiles of gene expression in the embryonic development of \textit{Drosophila} (I, Fig. 5).
%results
As expected, the most common profile (n=297 genes) follows the global profile of non-linear decrease in the first stages (I, Fig 5).

Among the rest of profiles, I found both linear increase and decrease profiles and a `hill-like' profile (initial increase and further decrease with the higher values at stage 7-8)
%
The linear decrease profile (n=167 genes) was enriched with `mitotic cell cycle' (GO:0000278), `RNA processing' (GO:0006396) and `chromatin modification' (GO:0016568) GOterm genes, highlighting biological processes that first are present in the whole embryo and become more and more restricted in space as development proceeds.
%discussion
The `mitotic cell cycle' term, for example, most likely relates to the fast mitotic cycles in the earliest embryo. During stage 1-3 nine fast and synchronic mitotic divisions take place in the entire embryo, then in stage 4-6 mitotic divisions 10-13 occur more slowly, almost synchronically. The 14th cycle, zygotically controlled, is long and of different durations in the embryo.

With a temporal co-expression cluster analysis using microarray data through the life cycle of \textit{D. melanogaster}, \citet{Arbeitman2002} found that most cell cycle genes were expressed at high levels during the first 12h, but only a few are expressed at high level thereafter.
My analysis is consistent with this, as I found that the profile of linear decrease (I, Fig. 5A) is enriched with such genes. In this sense, this study is complementary to Arbeitman et al., and adds the spatial dimension to their temporal expression profiles.
