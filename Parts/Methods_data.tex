The work presented here is mainly based on the analysis of publicly available data contained in many databases, introduced in the Review of the Literature section. In the next paragraphs I will describe briefly how the data used in here was acquired and processed. For more details see the corresponding study.

\subsection{In situ Hybridization data}

\subsubsection{\textit{D. melanogaster} (study I and III)}
\label{insitu_drosophila}
\paragraph{Image acquisition and filtering.}

Images were systematically downloaded (with an ad hoc Perl script) from FlyExpress version 5.1 \citep{Kumar2011} on February 2013. Only genes with laterally oriented images for the six stages used in BDGP \citep{Tomancak2002} were considered.

Images were resized as in \citet{Konikoff2012} and the gene expression pattern was extracted using an adaptive threshold, based on the mean and variance of a grey-scale version of each image. Genes with ubiquitous expression in stages 1-3 and 4-6 were considered as entirely black images.
To correct for small variations in the shape of the embryos, I morphometrically deformed each embryo to an stage ideal embryo shape. Finally, I applied a "smoothing filter" and eliminate isolated white/black pixels (see study I for details). 
I manually filtered images from the literature or directly from BDGP of Transcription Factors or Growth Factor genes that did not have information in FlyExpress.
 
For study I, the resulting dataset contained 1218 genes with expression information in the six stages used in BDGP. For study III, obsolete or repeated genes were removed from the 1218 genes, after checking for gene annotation updates with the \textit{biomaRt} package \citep{Durinck2009}, leaving a dataset of 1199 genes.

\paragraph{Main spatio-temporal expression profiles (study I).}
I performed a time series cluster analysis (with the STEM software; \citealp{Ernst2006}) using the relative area of expression to know which are the most common spatio-temporal profiles. The resulting clusters were analysed with a GO term analysis with the same software.

\paragraph{Anatomical terms (study III).}
In addition to the whole-mount in situ mRNA hybridization images, the BDGP database contains, for each gene, the list of the embryonic anatomical structures in which such gene is expressed \citep{Tomancak2007}. Each gene expression is described by one or several of those of anatomical terms by an expert. 
This information was retrieved from the BDGP downloads page (http://insitu.fruitfly.org/ insitu/html/ downloads.html/), which contains the annotations of almost 8,000 genes. We removed genes with "no staining" as anatomical term in any stage, leaving a total of 5762 genes.

\subsubsection{\textit{C. intestinalis} (study II)}

I downloaded the in situ hybridization data (ish.zip file) from the download section of the ANISEED database on 28th of December 2015. 
The expression data for the first three stages is at the cell level, while in the tailbud stages is at the tissues or specific regions of the embryo level. 
I extracted the information of the 32 cells, 64 cells, 112 cells, early tailbud, mid tailbud and late tailbud stages. Only expression data from experiments reported to have Wild type phenotype, "public" publication status, with in situ hybridization as experiment design and whose probe was assigned to a Kyoto Hoya (KH) \citep{Satou2008} gene model.
I excluded data from experiments whose image characterization was reported as "not sure" or too broadly as "part of whole embryo".

The number of genes analyzed is n=745 for the 32-cell stage, n=758 for the 64-cell stage, n=809 for the 112-cell stage, n=1082 for the early tailbud, n=1092 for the mid tailbud and n=887 for the late tailbud. 
The gene expression as text-based annotation was transformed into a 3D expression pattern using 3D embryo models (see below) with the Meshlab software version v1.3.3\_64bit (Meshlab Visual Computing Lab ISTI-CNR; see study II for details).

\paragraph{Transcription factors and signalling genes}
I used the comprehensive list of TFs (http:// ghost.zool.kyoto-u.ac.jp/ TF\_KH.html) and SIGs (http:// ghost.zool.kyoto-u.ac.jp/ ST\_KH.html) deposited in the Ghost database (last access in July 2015).
 
This list is based mainly in \citet{Imai2004}, who determined the expression profiles of 389 transcription factors (TFs) and 118 signaling molecules (SIGs) genes from the egg to mid-tailbud embryos. TFs are categorized in nine gene families: basic helix-loop-helix (bHLH), homeodomain (HD), Fox, ETS, bZIP, nuclear receptor (NR), HMG, T-box transcription factors or as "other TFs" (mainly with diverse Zinc finger genes).
The SIGs genes consist of genes of receptor tyrosine kinase pathways including ligands such as FGFs and intracellular signalling molecules such as MAPK, Notch, Wnt, TGF$\beta$, Hedgehog and genes in the JAK/STAT pathways. 

\paragraph{3D embryo models}

I downloaded, from the ANISEED database, 3D embryo models (at a single-cell resolution) for the 32-cell, 64-cell and 112-cell stages. Also for these stages, I downloaded files (biometry.zip folder; see study II) with a quantitative description of the geometry of individual blastomeres, including the volume of each blastomere relative to the whole embryo \citep{Tassy2006} used in the relative volume analysis.
For the tailbud stages I used a 3D model of \textit{C. intestinalis} mid tailbud (stage 22) anatomy at a single cell resolution \citep{Nakamura2012}, downloaded as a file "3DVMTE\_THratio1.86.wrl" from http://chordate.bpni.bio .keio.ac.jp/3DVMTE/. 3D embryo models for early and late tailbud were not available, so I used the 3D mid tailbud for all the tailbud stages. In these stages the main morphogenetic process is the tail elongation by cell intercalation \citep{Hotta2007}, so the differences between these stages are largely restricted to tail length and width and should not affect largely the relative volume measure.
From this file, I manually extracted the information of different tissues into separate 3D files and processed them using diverse filters of the Meshlab software version v1.3.3 (Meshlab Visual Computing Lab ISTI-CNR; see study II).


\subsection{Transcriptomics and population genomic data}

\paragraph{modENCODE (study III and IV).}

Gene expression levels in reads per kilobase per million mapped reads (RPKM) units for 30 developmental stages were retrieved from \citet{Gelbart2013}, who analyzed RNA-seq throughput data from the modENCODE project \citep{Graveley2011}.

For using RNA-seq data to compare expression between samples, a normalization step was performed to adjust for varying sequencing depths and other potential technical effects across replicates (see study III)

\paragraph{DGRP (study III and IV).}

The population genomic data comes from 168 inbred lines of \textit{D. melanogaster} sequenced in the Freeze 1.0 of the Drosophila Genetic Reference Panel (DGRP) project \citep{Mackay2012}. The DGRP population was created collecting gravid females from a single population of Raleigh, North Carolina (USA), and following the full-sibling inbreeding approach during 20 generations to obtain full homozygous individuals. 
DGRP lines showing high values of residual heterozygosity (>9\%) that were observed to be associated with large polymorphic inversions \citep{Huang2014} were not included.

\paragraph{Testes and immune genes (study III).}
To discard the possibility that the adaptation patterns are due to an excess of male-biased genes, testes specific genes or immune-related genes, known of being under positive adaptation \citep{Civetta1995,Swanson2001,Artieri2009,Obbard2009}, genes related to these functions were removed (for details see Methods study III).

\paragraph{Maternal, maternal-zygotic and zygotic genes (study III).}
A list of maternal, semi-maternal and zygotic genes was obtained using data from \citet{Thomsen2010}, who performed microarray analyes of unfertilized eggs and the early zygote embryos (for details see Methods study III).