Measuring adaptation is an important topic in evolutionary biology.
Since Darwin.. (REF Darwin)
In here, I refer to adaptation as a phenotypic character (or modification in a phenotypic character) that arise by natural selection in response to the environment, or other external factor.
Measuring adaptation at the phenotypic level requires a clear understanding of the function of the phenotypic character under study, and how the modification of this trait would affect the fitness of the bearer organism 
	\citep{EmiliaSantos2015}. 

Importantly, all phenotypic changes (whether a new character or a modification of an existing character) is produced from a change in development.
For example, the difference in the beak size and shape between the famous Galapagos Darwin's finches (REF Darwin), a classic example of adaptive change under natural selection, has been shown to be regulated by the differential expression of the genes CaM 
	\nomenclature{CaM}{Calmoduline}
and BMP4 
	\nomenclature{BMP4}{Bone morphogenetic protein 4}
during development. A proposed model for BMP4 and CaM role in beak size and shape explains both elongated and deep/wide beaks of these finches
	\citep{Abzhanov2006}.

So, even when natural selection acts in the adult phenotype (or in larva, in the case of species with a feeding larva stage), we should be able to find changes in development that would explain an adaptive change in the adult or larva.


Some other remarkable examples of the genetic-developmental basis of adaptive change are:

-

-

It is evident at this point than many of the developmental changes leading to an adaptation are (at least partially) caused by mutations in gene regulatory or coding sequences.
Therefore, the effects of natural selection could be traceable looking at the adaptive changes in the genes expressed in different times and locations during development. There is an entire field within evolutionary biology, namely molecular evolution, dedicated to explain the sequence changes in molecules as DNA,
	\nomenclature{DNA}{Deoxiribonucleic acid}
RNA 
	\nomenclature{RNA}{Ribonucleic acid}
and proteins. 


\subsection{Molecular evolution}

The theoretical basis of the molecular evolution field includes concepts from evolutionary biology and population genetics. At the DNA level, any transmissible change in the sequence is considered a mutation. 
The most simple change is a point mutation or single nucleotide polymorphism (SNP),
\nomenclature{SNP}{Single nucleotide polymorphism} 
which is a change in a single nucleotide in the DNA sequence of a locus of two individuals. 
If the individuals belong to the same species, this mutation is referred as polymorphism. In contrast, divergence refers to the mutations when individuals from different species are taken into account. 
SNPs occur in non-coding and coding DNA sequence. A single point mutation that occurs in a coding sequence can be classified in two categories, depending on the effect of this mutation in the protein sequence: i) synonymous mutation and ii) non-synonymous mutation.
A synonymous mutation does not affect the amino-acid sequence of the protein, albeit it can affect its function 
	\citep{Kimchi-Sarfaty2007}
or the gene transcriptional efficiency (REF).
A non-synonymous mutation does affect the amino-acid sequence of the protein whether by changing a single amino-acid (missense mutation) or by producing a stop codon (non-sense mutation) which results in a truncated version of the protein.


An important topic within the molecular evolution field is the identification of DNA sequence adaptive changes in a species. 
For doing this, many statistical tests have been developed. 
Importantly, these tests are based on the neutral theory of evolution, proposed by Kimura
	\citep{Kimura1968}.
One of the most popular is the McDonald-Kreitman test (MKT),
	\nomenclature{MKT}{MacDonald-Kreitman test}
which estimates the proportion of the adaptive substitution resulted from natural selection. 
Recently, more sophisticated methods based on the MKT, like the DFE test (REF Eyre), have been developed to correct for underestimation of adaptive evolution in the presence of slightly deleterious mutations.

\subsubsection{Neutral theory of evolution}
In 1968, Mooto Kimura calculated the average rate of nucleotide substitutions in the evolutionary history of mammals.
The result of his calculations was that, on average, one nucleotide has been substituted every 2 years.
For him, this very high rate of substitution was only explainable if most mutations were almost neutral in natural selection 
	\citep{Kimura1968}.
This was in contrast with the prevailing view at the time that practically no mutations are neutral (REF).
More importantly, the neutral theory provided a set of testable predictions, providing a null-hypothesis of molecular evolution.
This allowed the development of statistical methods to detect adaptive changes, i.e., we can say that a sequence has been under positive selection if the amount of changes exceeds the number of changes expected only by neutral evolution.

\subsubsection{McDonald-Kreitman test}
John H. MacDonald and Martin Kreitman developed this test in 1991 to 


When positive or negative selection (natural selection) influences nonsynonymous variation, the ratios will no longer equal. 
The ratio of nonsynonymous to synonymous variation between species is going to be lower than the ratio of nonsynonymous to synonymous variation within species (i.e. Dn/Ds  Pn/Ps) when negative selection is at work, and deleterious mutations strongly affect polymorphism.

The ratio of nonsynonymous to synonymous variation within species is lower than the ratio of nonsynonymous to synonymous variation between species (i.e. Dn/Ds Pn/Ps) when we observe positive selection. 
Since mutations under positive selection spread through a population rapidly, they don't contribute to polymorphism but do have an effect on divergence.


\subsubsection{DFE test}


