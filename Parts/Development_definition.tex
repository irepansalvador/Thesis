It seems that there is no unique or straightforward answer to this question.
Sometimes, the study of development is implicitly considered to be the same as the the study of embryology \citep{Horder2010}.
%Traditionally, the problem of development has been studied by embryologists. However, embryonic development does not necessarily equate to development,
%theDevelopment is sometimes equate to embryonic development, probably due to the fact that developmental biology origins come from embryology.
%Equating embryonic development to development could be problematic 
This could be problematic when considering organisms with complex life cycles. For example, holometabolous insects, in addition to embryonic development, undergo a complete metamorphosis (from pupa to adult), a process that could be considered a second embryonic development.

Currently, the most common definition of development refers to the set of processes through which an egg is transformed into an adult \citep{Horder2010,Minelli2011}.
Already in 1880, Ernst Haeckel defined development in similar terms: "individual development, or the ontogenesis of every single organism, from the egg to the complete form is nothing but a growth attended by a series of diverging and progressive changes" \citep{haeckel_historycreation1880}.

Some authors criticize this egg-to-adult view to be an "adultocentric" view of development, and suggest instead to consider within the boundaries of development the whole life cycle of an organism \citep{Gilbert2011,Minelli2011}.
Julian S. Huxley and Gavin R. de Beer said that development "is not merely an affair of early stages; it continues, though usually at a diminishing rate, throughout life" \citep{huxley1963elements}.

%However, that this concept is not applicable to some organisms (Minelli in book). Some animals or plants instead of having an egg or seed stage as means of reproduction have buds or other vegetative parts.

%But even after considering the whole life cycle complications appear in cases in which the common notion of development of an individual organism would not apply, as in polyembryonic development or colonial organisms \citep{Minelli2011}.
There have been recent attempts to construct a broader concept of development \citep{Griesemer2014,Moczek2014,Pradeu2014} For example, Armin P. Moczek defines development as "the sum of all processes and interacting components that are required to allow organismal form and function, on all levels of biological organization, to come into being" \citep{Moczek2014}.
%
The main challenge on adopting a new concept of development which is more inclusive, is to maintain its intuitiveness and applicability in scientific research.

Throughout this dissertation I will use the "common view" of development \citep{Minelli2014}, that considers the egg and the adult as the start and end of individual development respectively.
%even when for practical reasons, some analysis of this dissertation (article 1 and 3) include only embryonic development. 
However, and mainly for practical reasons, the major part of the analysis presented here (articles 1-3) is (ARE?) restricted to embryonic development.
