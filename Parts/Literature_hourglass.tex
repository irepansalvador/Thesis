
%
%\subsubsection{Adaptation in the embryo}
%
%Usually, the study of adaptation
%- general considerations of adaptation in the embryo
%
%- example from marine animals
%- Whyte
%- entrenchment
%- adaptive changes that influence adult/larva characters
%- measurements of adaptation in the embryo in drosophila
%- heatshock, developmental time, Dn/Ds.. etc.
%\subsubsection{Hourglass model}

%%% von Baer vs Hourglass
\label{hourglass}

In the 19th century, Karl Ernst von Baer stated in his "laws" that within a group of animals the general characteristics appear earlier in development, while the most special appear in late development (see Box 1; \citealp{vonBaer1828uber}).
%although he recognized that the early stages were different (Gilbert 2010)
This would lead to low morphological variation at early development, gradually increasing as development proceeds.

Other authors \citep{Medawar1954,Slack1993,Duboule1994,Raff1996} proposed an alternative pattern in which there is great variation in early and late development, while the mid-development would show less variation.
This pattern of variation (or conservation) has been called `phylotypic egg-timer' \citep{Duboule1994} and `developmental hourglass' \citep{Raff1996}.

% hourglass background
Duboule's concept of `phylotypic egg-timer' was based in the concept of `phylotypic stage' of \citet{Sander1983}, who coined this term to describe the convergence into a conserved segmented germ band stage in insects from very divergent early development \citep{Sander1996}.
In vertebrates, there has been controversy around what should be the phylotypic stage \citep{Ballard1981,Slack1993,Duboule1994}. \citet{Richardson1995} argued that indeed there is no single conserved stage in vertebrate's development and instead he proposed the term `phylotypic period' instead.

Initially, two explanations for the hourglass model were proposed.
Denis Duboule, after observing that the expression of the Hox genes seemed to coincide with the phylotypic stage, he considered that this could not be a coincidence and proposed that the activation of the Hox genes was the cause for the morphological invariance \citep{Duboule1994}.
In contrast, Rudolf A. Raff proposed that the phylotypic stage was the result of complex interaction between developmental modules at this stage  \citep{Raff1996}.

There is an ongoing discussion about whether the hourglass model (HG), the von Baer law
or some other pattern fits the divergence among developmental stages in phylogeny \citep{Richardson1997,Poe2004,Kalinka2012}.
	\nomenclature{HG}{Hourglass model}

Also, it is not clear if the HG, that seems to fit well in vertebrates and arthropods, would apply to other phyla \citep{Raff1996} \citep{Salazar-Ciudad2010}. For example, it is known that the HG model does not apply to spiralians, as many members of this phyla exhibit an early equal cleavage pattern \citep{Henry2002}.
%\citet{Salazar-Ciudad2010} has proposed that different patterns of variation throughout development in metazoan groups would correlate with different developmental types (a classification based on the relative use of signalling and morphogenetic events).

%% genetic hourglass
Recently, the HG have received support from different gene expression studies.
%Kalinka
\citet{Kalinka2010} used micro-arrays for six Drosophila species and quantified expression divergence at different developmental stages. They found that gene expression was most conserved during the extended germ-band stage (considered the phylotypic period) and that the non-synonymous divergence per site ($Dn$) correlated with their divergence measures.

They also proposed that the HG pattern is a product of natural selection that acts to conserve patterns of gene expression during mid-embryogenesis \citep{Kalinka2010}.
%
%Domazet & piasecka
Also, the HG model has been shown to be reflected in the age of the transcriptome. \citet{Domazet-Loso2010} found that when analysing the age of the transcriptome at different stages in the Zebra fish (\textit{Danio rerio}) development (analyzing gene-specific expression data with a phyostatigraphic method), mid-embryonic stages show the older transcriptome \citep{Domazet-Loso2010}.
In another analysis using Zebra fish, it was shown that the mid-development conservation included that of regulatory regions, with sequences of regulatory regions being most conserved for genes expressed in mid-development; \citealp{Piasecka2013}).

%other based on DNA conservation
Studies that have measuring the conservation of genes at the DNA sequence level also seem to support the HG model.
\citet{Davis2005} assessed whether proteins expressed at different times during \textit{D.  melanogaster} development varied systematically in their rates of evolution (comparing with \textit{D. pseudoobscura}) and found that proteins expressed early in development and particularly during mid-late embryonic development evolve slower.  
This suggests, according to the authors, that embryonic stages from 12 to 22 hours are
highly conserved between \textit{D. melanogaster} and \textit{D. pseudoobscura}, which is consistent with the HG.
%
In a similar study, \citet{Mensch2013} calculated the dN/dS ratio for more than 2,000 genes among six \textit{Drosophila} species, separating genes in three categories: maternal genes (genes whose products are left by the mother in the egg), genes expressed in early development and genes expressed in late development. They found that maternal genes and lately expressed zygotic genes show higher dN/dS ratios (i.e., are less conserved) than early expressed zygotic genes.
Finally, it has also been found that genes expressed in the adult have higher dN/dS ratios than genes expressed in the pupa and those of the pupa have higher dN/dS ratios that those expressed in the embryo \citep{Artieri2009}.

Some limitations of these last studies is that they classify the genes in a few broad temporal categories that do not permit to precisely determine the temporal dynamics of conservation and that are based only in divergence data (dN/dS ratios between two species).
A study that integrates polymorphism data from natural populations would improve the evolutionary interpretation of these patterns, as it would allow to estimate what proportion of the dN are adaptive (as explained in section \ref{alpha}).
Measuring adaptation is specially relevant as some authors have argued that the HG is caused by different selection pressures in early and late development \citep{Slack1993,Kalinka2012,Wray2000}

%%inter phyla comparison

%Recently, a study reported an inverted hourglass pattern
%(i.e., early and late conservation and mid-development divergence) when comparing developmental transcriptomes of ten different species, each from a different phyla [21](REF).
\clearpage
%%%%%%%%%%%%%%%%%%%%%%%%%%%%%%%%%%%%%%%%%%%%%%%%%%%%%%%%%%%%%%%%%%%%%%%%%%%%%%%%%%%
\begin{mdframed}[style=boxstyle,frametitle={Box2. Haeckel, von Baer and the \textit{Naturphilosophie}}]\label{Box2:Naturphilosophie}

In the 19th century, important contributions to embryology were made by advocates of \textit{Naturphilosophie}, a philosophical movement based in Kant and Goethe's ideas, aimed to classify nature into categories or classes.
Among their classification efforts, they classified embryological phenomena and draw analogies between embryos of different taxonomic groups \citep{Horder2010,Ghiselin2005}.

The first pattern to be recognized, when comparing developmental trajectories of different species, was the Meckel-Serres law,
% (named after two of their main proponents: \'{E}tienne Serres and Johann Friedrich Meckel; \citealp{Russell1916}),
  which proposed that embryos followed a linear succession following the \textit{scala naturae} (a hierarchy of all beings arranged in order of `perfection', with the man at the top).
According to this view, influenced by the \textit{Naturphilosophie}, the embryonic development of a higher organism would be a succession of adult forms of lower organisms \citep{Russell1916,amundson2005changing}.

\paragraph{Karl Ernst von Baer} \label{vonBaer}
K. E. von Baer, a German-Estonian naturalist considered the father of comparative embryology \citep{Russell1916}, refuted the Meckel-Serres law and formulated his own, known as von Baer's laws \citep{vonBaer1828uber}. 
Von Baer's first law state that the more general characteristics of a large animal group (e.g., notochord in chordates) develop before special characteristics (e.g., fur in mammals), while his fourth law state that the embryo of a "higher" animal never resembles the adult of another animal form, but only his embryo. 

Importantly, von Baer's views were not evolutionary. The resemblance between developmental trajectories of different species was for him only a reflection of their relationship in the Natural System \citep{amundson2005changing}.
Ironically, Darwin used and reinterpreted von Baer's observations on embryonic stages in different species to support common ancestry and therefore, evolution \citep{darwin1859origin}.

\paragraph{Ernst Haeckel}
Ernst Haeckel supported Darwinism and, in what is known as Haeckel's "Biogenetic Law", said that development (or ontogeny) is a brief summary of the slow and long phylogeny \citep{haeckel1874menschen}.
In his view, a "higher" organism would pass through a series of conserved developmental stages that represent ancestral forms (this view is known as the "recapitulation theory").
However, in contrast with the Meckel-Serres law, he recognized that this recapitulation was almost never complete, due to evolutionary modifications in development. 
%He also classified two types of change in development, "heterochrony" and "heterotopy", concepts introduced by him that since then have been crucial in many discussions on the relationship between development and evolution \citep{Horder2013}:
%%%%%%%%%%%%%%%%%%%%%%%%%%%%%%%%%%%%%%%%%%%%%%%%%%%%%%%%%%%%%%%%%%%%%%%%%%%%%%%%%%%%%%%%%%%%%
\begin{flushleft}
\leftskip3em
\rightskip\leftskip
\footnotesize{
\textit{"The falsification of the original course of development is based to a great extent on a gradually occurring displacement of the phenomena, which has been effected slowly over many millennia, by adapting to the changed conditions of embryonic existence. This displacement can affect both their location and time of appearance. Those former we call heterotopy, the latter heterochrony." \citep{haeckel1903anthropogenie}.}}
\end{flushleft}
%%%%%%%%%%%%%%%%%%%%%%%%%%%%%%%%%%%%%%%%%%%%%%%%%%%%%%%%%%%%%%%%%%%%%%%%%%%%%%%%%%%%%%%%%%%%%
Haeckel's views were more complex than usually acknowledged \citep{Richardson2002}.
In fact, he said that it was not that all the mammalian eggs were the same, it was just that with the available tools was impossible to detect the subtle, individual differences, "which are to be found only in the molecular structure" \citep{haeckel1903anthropogenie}.

Now is evident that none of von Baer's or Haeckel's hypothesis can be considered "laws", as they are not universal.
%They only apply to some characters, stages and levels of phylogenetic inclusiveness \citep{Richardson2002}. 
%Nevertheless, the works of both Haeckel and von Baer represented the foundations of the comparative embryology field, which is in turn one of the basis of the modern evolutionary developmental biology (evo-devo).


\end{mdframed}
%%%%%%%%%%%%%%%%%%%%%%%%%%%%%%%%%%%%%%%%%%%%%%%%%%%%%%%%%%%%%%%%%%%%%%%%%%%%%%%%%%