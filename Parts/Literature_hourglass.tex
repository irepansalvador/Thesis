

\subsubsection{Adaptation in the embryo}

- general considerations of adaptation in the embryo

- example from marine animals

- Whyte

- entrenchment

- adaptive changes that influence adult/larva characters

- measurements of adaptation in the embryo in drosophila

- heatshock, developmental time, Dn/Ds.. etc.


\subsubsection{Hourglass model}

As we described in section X, von Baer stated in his "laws" that within a group of animals the general characteristics appear earlier in development, while the most special appear in late development \citep{vonBaer1828uber}.
This would lead to low morphogical variation at early development, gradually increasing as development proceeds.

Other authors \citep{Medawar1954,Slack1993,Duboule1994,Raff1996} proposed an alternative pattern in which there is great variation in early and late development, while the mid-development would show less variation.
The less variable (or more conserved) stage in mid development has been called the "phylotypic stage" or "phylotypic period".


although he recognized that the early stages were different (Gilbert 2010)

the hourglass model, proposed by (REFs)

mid-development conserved

natural selection?

Hox exagerations

With molecular evolution methods (see Subsec X), 

