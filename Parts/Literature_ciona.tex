
	The ascidian \textit{Ciona intestinalis}, a marine invertebrate animal, has a long history in developmental biology and evolutionary biology. 
	Darwin highlighted the importance of the ascidians due to their close phylogenetic relationship to the vertebrates \citep{Darwin20009}. Although their adult form is a sessile filter feeder, its tadpole larva has characteristic features of the chordate group: a dorsal neural tube, a notochord surrounded by muscle and a ventral endodermal strand \cite{Satoh2003}. Also, it provided one of the first evidences of localized determinants of cell specification \citep{Conklin1905}.
	
	Some features of \textit{C. intestinalis} development that attracted the attention of developmental biologists more than a century ago included: the rapid embryonic development (it takes less than 20 hours from the fertilized egg to the larva), its invariant cell lineage and similitude between the ascidian larva and the vertebrate tadpole \citep{kowalewski1866entwicklungsgeschichte,chabry1887contribution}.
	More recently, the almost transparent body, which facilitate many genetic techniques, and the sequencing of the \textit{C. intestinalis} genome \citep{Dehal2002} are partly responsible for the re-emergence of \textit{C. intestinalis} as model organism in developmental biology \citep{Levin2012}.
	
Relevant efforts have been made to describe the spatial expression patterns of individual genes (REF).The spatial expression patterns of  >1,000 cDNA clones have been described using whole-mount in situ hybridization techniques at different developmental stages \citep{Imai2004}. Importantly, the developmental stages included cover a wide temporal range, e.g., blastula, gastrula and tapole stages \citep{Satou2005}.
	 Taking advantage of the ascidian invariant cleavage pattern and well described lineage analysis \citep{Conklin1905,Nishida1987}, the spatial expression of many genes have been described at the single cell level up to the early gastrula stage (REF), making this an invaluable resource to investigate the spatio-temporal dynamics of gene expression.



%	The \textit{C. intestinalis} genome is only 160Mb and contains ~16,000 genes, a gene number similar to the invertebrate \textit{D. melanogaster} genome and only is half of the genes found in some vertebrates (REF).
%	This low number of genes (compared to vertebrates) can be explained by the finding that many gene families or subfamilies have only one representative in \textit{C. intestinalis} \citep{Dehal2002}.

\subsection{\textit{Ciona intestinalis} life cycle}


\subsubsection{Developmental stages}

Ascidians, or tunicates (named after the "tunic" or thick cover in the adult form), are sessile animals that  attach to rocks and shells and filter plankton and other nutrients from seawater \citep{satoh2014developmental}.
During embryogenesis, ascidians show morphogenetic movements during gastrulation and neurulation similar to vertebrates and both share common genetic regulators of cell specification \citep{Satoh2003}.

As noted above, its larval form resembles a vertebrate tadpole.


Quick intro of developmental stages at 18$^\circ$C, based on \citep{Hotta2007}.

%%%%%%%%%%%%%%%%%%%%%%%%%%%%%%%%%%%%%%%%%%%%%%%%%%%%%%%%%%%%%%%%%%%%%%%%%%%%%%%%%%%%%
\label{Table_Ciona}
\begin{sidewaystable}
    \centering

\caption*{\textbf{Table 2. Embryonic stages of \textit{C. intestinalis} and main morphological characteristics (based on \citealp{Hotta2007}) }}
\begin{tabular}{|c|p{3cm}|p{3cm}|p{10cm}|}
\hline
\textbf{Stage}&\textbf{Description}&\textbf{Time after fertilization}& \textbf{Morphological characteristics} \\
\hline
\textbf{1}	& \textbf{Zygote}		& 0min to 55min 	&	From the fertilisation event up to the end of the first mitotic cycle \\
%
\textbf{2-5}	& \textbf{Early cleavage}& 55min to 3h	& Five mitotic divisions, until the 32-cell stage. First and second cleavages separate the left and right halves, and the anterior and posterior halves, respectively.  
%Asymmetric division of B5.2 due to the action of the centrosome atracting body (CAB), leads to a small posterior B6.3 cell. 
\\
%
\textbf{6-9}	& \textbf{Late cleavage}& 3h to 4.5h	& Very small B7.6 cell pair in the posterior end. Asymmetric divisions in the vegetal hemisphere. Embryo flattens on its vegetal side. Vegetal cells take a columnar shape\\
%
\textbf{10-13}	& \textbf{Gastrula}	& 1.5h to 6.3h	& Invagination and migration of endodermal and mesodermal cells inside the embryo. At the early gastrula, the vegetal side of the embryo takes a horseshoe shape. Embryo starts elongating anteriorly \\
%
\textbf{14-16}	& \textbf{Neurula}	& 6.3h to 8.5h	& The embryo, with an oval shape, continues elongating. Notochord precursors  intercalation and convergence. Neural tube formation and closure (starting from the posterior side)\\
%
\textbf{17-20}	& \textbf{Initial and early taibud}	& 8.5h to 10h	& Clear separation between trunk and tail. Neuropore closure. Tail starts bending \\
%
\textbf{21-22}	& \textbf{Mid tailbud}	& 10h to 11.9h	& Intercalation of the notochord cells is completed. The tail bends ventrally so the embryo adopts a half-circle shape. Length of the tail twice as long as the trunk	\\
%
\textbf{23-25}	& \textbf{Late tailbud}	& 11.9h to 17.5h 	& Pigmentation of the otolith can be observed. Palps formation. Vacuolization of notochord cells. The tail is bent dorsally   \\
%
\textbf{26}		& \textbf{Hatching}	& 17.5h 	& The larva hatches. Head adopts an elongated rectangular shape \\

\hline
\end{tabular}
%\end{adjustwidth}
\end{sidewaystable}

\subsection{The ANISEED database}

The ANISEED database (2015 version) integrates expression data from large-scale in situ hybridization studies with embryo anatomical data of ascidians \citep{Tassy2010,Brozovic2016}. 
This database includes 27,707 \textit{Ciona intestinalis} gene expression profiles by in situ hybridisation for approximately 4500 genes acquired from more than 200 manually curated articles \citep{Brozovic2016}.

The expression data is represented using an ontology-based anatomic description of the embryos. The ANISEED database also includes expression data from the Ghost database in \textit{Ciona intestinalis} \citep{Satou2005}, which contains the spatial expression patterns of  >1,000 cDNA clones by whole-mount in situ hybridization at different developmental stages.
Taking advantage of the invariant ascidian cleavage pattern and well-described lineage analysis \citep{Conklin1905,Nishida1987} the cDNA spatial expression have been described at the single cell level up to the early gastrula stage \citep{Imai2004}.

The ANISEED database also includes biometry data (e.g., volume, surface/volume) and 3D embryo models (at a single-cell resolution) of ascidian embryos until the early gastrula \citep{Tassy2006}. Importantly, combining the gene expression and 3D embyo models at a single cell resolution it is possible to reconstruct the gene expression pattern in 3D, as I did here. 
As the ANISEED database do not contains 3D embryo models for tailbud stages, in order to do reconstruct the gene expression in 3D in these stages, I used a 3D model of \textit{C. intestinalis} mid tailbud anatomy at a single cell resolution \citep{Nakamura2012}(see methods and study II).
