The statistical approach I have used in here, is nothing but new.

Darwin used a statistical approach to describe the action of natural selection (REF Darwin). For him, given the origination of small variations in natural populations, the occurrence of any advantageous variation in an individual, as slight it could be, would be reflected in a better chance of survival and to procreating their kind (Darwin). With many generations, the differential survival of the variants, would produce a change in the population mean. 
The effects of natural selection are thus only observable at the population level.

A more formal approach came from physics, more precisely from the study of diffusion of gases in the 19th century.

Against the main views of his contemporaries, which considered that all the particles in a gas move at the same speed, J. C. Maxwell proposed that each particle of a gas moved with different velocity and direction, both changing after the particles collision among them (REF Maxwell 1,2).
The velocities in all directions are distributed among the particles according to a certain law. As it was impossible to observe the behaviour of all the particles, their properties could only be described at a statistical level, as the average movement of large numbers of gas particles.

For Boltzmann and Gibbs, which extended the studies on gas diffusion, the study of large numbers was not only important to overcome the problem of not being able to study each individual particles, also because their individual behaviour is not interesting at all (Jacob, logic of life). Knowing the movement and direction of each particle would not give more information than the population as a whole.

After the success of statistical mechanics, its methodology expanded to many other scientific fields.
Laws could be applied to solve previously intractable problems by collecting sufficient information of a great number of cases of the same class and calculating its mean. The aim of the statistical approach is then to "obtain a law which transcends individual cases" (Jacob).

This novel approach changed biology drastically, transforming it into a quantitative science. As Fran\c{c}ois Jacob said, "at the end of the nineteenth century, the study of living beings was no longer a science of order, but one of measurement as well".

