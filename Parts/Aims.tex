
In this work, I have analysed publicly accessible spatio-temporal gene expression data of two model organisms, \textit{Drosophila melanogaster} and \textit{Ciona intestinalis}, together with population genomics data of \textit{D. melanogaster}.
Using a statistical approach, I address three questions:
%%%%%%%%%%%%%%%%%%%%%%%%%%%%%%%%%%%%%%%%%%%%%%%%%%%%%%%%%%%%%%%%%%%%%%%%%%
\begin{enumerate}
\item How do complexity and compartmentalization increase in the embryo during development?
\item Can adaptation be found in specific anatomical parts of the embryo or developmental stages? 
\item Is the Hourglass model (which states that there is less amount of inter-specific variation in mid-development; see section \ref{hourglass}) supported by evidence of natural selection at the DNA sequence level?
\end{enumerate}
%%%%%%%%%%%%%%%%%%%%%%%%%%%%%%%%%%%%%%%%%%%%%%%%%%%%%%%%%%%%%%%%%%%%%%%%%%

These questions have been selected for the great interest they have aroused in the scientific community since the early days of developmental and evolutionary biology.
%
The work presented here is based on and uses concepts from three main biology fields: 
developmental biology, evolutionary biology and population genetics.
Nowadays, the combination of these scientific fields constitute multiple research programmes. Modern evolutionary developmental biology (evo-devo) is the explicit combination of the first two fields.
	\nomenclature{evo-devo}{Evolutionary developmental biology}
