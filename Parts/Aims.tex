\vspace{1cm}
In this work, I have analysed publicly accessible spatio-temporal gene expression data of two model organisms, \textit{Drosophila melanogaster} and \textit{Ciona intestinalis}, together with population genomics data of \textit{D. melanogaster}.
Using a statistical approach, I address these following questions, which have been selected for the great interest they have aroused in the scientific community since the early days of developmental and evolutionary biology:

\vspace{1cm}
%%%%%%%%%%%%%%%%%%%%%%%%%%%%%%%%%%%%%%%%%%%%%%%%%%%%%%%%%%%%%%%%%%%%%%%%%%
\begin{enumerate}[label=\Roman*]

\item How do complexity and compartmentalization increase in the embryo during development?
\vspace{0.5cm}

\item Are there differences in the pattern of compartmentalization and complexity increase when comparing different species (i.e., \textit{D. melanogaster} and \textit{C. intestinalis})?
\vspace{0.5cm}

\item Can adaptation be found in specific anatomical parts of the embryo or developmental stages? 
\vspace{0.5cm}

\item Is the Hourglass model supported by evidence of natural selection when considering inter and intra-specific variation at the DNA sequence level?

\end{enumerate}
%%%%%%%%%%%%%%%%%%%%%%%%%%%%%%%%%%%%%%%%%%%%%%%%%%%%%%%%%%%%%%%%%%%%%%%%%%
\vspace{1cm}

%
%These questions have been selected for the great interest they have aroused in the scientific community since the early days of developmental and evolutionary biology.
%%
%The work presented here is based on and uses concepts from three main biology fields: 
%developmental biology, evolutionary biology and population genetics.
%Nowadays, the combination of these scientific fields constitute multiple research programmes. Modern evolutionary developmental biology (evo-devo) is the explicit combination of the first two fields.
%	\nomenclature{evo-devo}{Evolutionary developmental biology}
