
%This work is based and uses concepts from three main biology fields: 
%developmental biology, evolutionary biology and genetics.
%
%%
%Nowadays, the union of these scientific fields form multiple research programmes. 
%Only in evolutionary developmental biology (evo-devo), the explicit union of the first two fields, at least four major research programmes have been recognized (Muller, 2006).
%However, these fields have not always gotten along well. 
%Some decades ago, there was a (CLEAR? CONCEPTUAL, EPISTEMOLOGICAL?) separation between evolutionary biology and developmental biology, even when embryology (which slowly transformed into developmental biology in the middle of the 20th century, see \citealp{Horder2010}) was considered crucial for the study of evolution in the 19th century.
%
%In the following section, I will make a brief introduction of the scientific (and philosophical) origins of developmental biology, with special attention to its relations with evolutionary biology and genetics (for a comprehensive review on this issue, see: \citealp{amundson2005changing}; \citealp{gilbert1991conceptual}).
%
%But before that, it might be useful to define what development is, so firstly, I will address this apparently simple question.

%This is not intended to be a comprehensive review, for a more complete view see \citep{amundson2005changing,gilbert1991conceptual}.

%Thesis main subjects:
%increase of complexity in development with an evolutionary perspective
%adaptation in the embryo

%
%\subsection{What is Development?}
%
%It seems that there is no unique or straightforward answer to this question.
%Sometimes, the study of development is implicitly considered to be the same as the the study of embryology \citep{Horder2010}.
%%Traditionally, the problem of development has been studied by embryologists. However, embryonic development does not necessarily equate to development,
%%theDevelopment is sometimes equate to embryonic development, probably due to the fact that developmental biology origins come from embryology.
%%Equating embryonic development to development could be problematic 
%This could be problematic when considering organisms with complex life cycles. For example, holometabolous insects, in addition to embryonic development, undergo a complete metamorphosis (from pupa to adult), a process that could be considered a second embryonic development.
%
%Currently, the most common definition of development refers to the set of processes through which an egg is transformed into an adult \citep{Horder2010,Minelli2011}.
%Already in 1880, Ernst Haeckel defined development in similar terms: "individual development, or the ontogenesis of every single organism, from the egg to the complete form is nothing but a growth attended by a series of diverging and progressive changes" \citep{haeckel_historycreation1880}.
%
%Some authors criticize this egg-to-adult view to be an "adultocentric" view of development, and suggest instead to consider within the boundaries of development the whole life cycle of an organism \citep{Gilbert2011,Minelli2011}.
%Julian S. Huxley and Gavin R. de Beer said that development "is not merely an affair of early stages; it continues, though usually at a diminishing rate, throughout life" \citep{huxley1963elements}.
%
%%However, that this concept is not applicable to some organisms (Minelli in book). Some animals or plants instead of having an egg or seed stage as means of reproduction have buds or other vegetative parts.
%
%%But even after considering the whole life cycle complications appear in cases in which the common notion of development of an individual organism would not apply, as in polyembryonic development or colonial organisms \citep{Minelli2011}.
%There have been recent attempts to construct a broader concept of development \citep{Griesemer2014,Moczek2014,Pradeu2014} For example, Armin P. Moczek defines development as "the sum of all processes and interacting components that are required to allow organismal form and function, on all levels of biological organization, to come into being" \citep{Moczek2014}.
%%
%The main challenge on adopting a new concept of development which is more inclusive, is to maintain its intuitiveness and applicability in scientific research.
%
%Throughout this dissertation I will use the "common view" of development \citep{Minelli2014}, that considers the egg and the adult as the start and end of individual development respectively.
%%even when for practical reasons, some analysis of this dissertation (article 1 and 3) include only embryonic development. 
%However, and mainly for practical reasons, the major part of the analysis presented here (articles 1-3) is (ARE?) restricted to embryonic development.
%
%\subsection{On the history of Developmental Biology}

In the last decades the scientific community has witnessed the flourishing of developmental biology.
Since the 1980's crucial discoveries \citep{Gilbert1998} have not only improved our understanding of the developmental process, but also changed the perspective of the explanatory role of development in biology.

However, developmental biology can not be considered a young scientific discipline, as its roots come from centuries ago, back from embryology and anatomy.
%The roots of the modern development biology come from embryology and anatomy.
%In the next paragraphs I make a brief introduction of the scientific (and philosophical) origins of developmental biology.
%In the following paragraphs, I will attempt to briefly describe the history of developmental biology, which 
%The history of developmental biology can be divided in four main periods: from Aristotle to the 18th century, the preformationism-epigenesis debate in the 18th century, the study of "developmental mechanics" in the 19th century and the molecular genetics period together with the emergence of evolutionary developmental biology (evo-devo).
The summary presented here grasps only the surface of this history, for further and deeper lecture, see \citep{gilbert1991conceptual,amundson2005changing,hall1999evolutionary} %(Jacob?)


\subsection{Aristotle}
%which until the beginning of the 19th century was mostly a descriptive science.
Before the 19th century, the major single contributor in the study of embryology was Aristotle. 
Some of his most important contributions to embryology are:

%%%%%%%%%%%%%%%%%%%%%%%%%%%%%%%%%%%%%%%%%%%%%%%%%%%%%%%%%%%%%%%%%%%%%%%%%%%%%%%%%%%%%%%
\begin{enumerate}
\item He organized and classified animals accordingly to their embryonic development after careful observation of the development in many species \citep{aristotle1979generation}. Because of this, he can be considered the first comparative embryologist \citep{Needham1959}.
\item For him, any developmental process was driven by "internal causes" that required a "soul" to guide it.
\item He clearly defined two opposite theories of development, preformationism and epigenesis, from which he supported the latter.
\end{enumerate}
%%%%%%%%%%%%%%%%%%%%%%%%%%%%%%%%%%%%%%%%%%%%%%%%%%%%%%%%%%%%%%%%%%%%%%%%%%%%%%%%%%%%%%%

After Aristotle, the preformationism-epigenesis debate would last centuries attracting many of the most important philosophers and naturalists.

\subsection{The preformationism-epigenesis debate}

Until the 18th century, supporters of epigenesis (like Wolffs and ??) saw development as starting from a formless embryo, with its form arising following a "vital" force \citep{amundson2005changing}.
%
During the 18th century, however, many rejected any vital force to explain development, leaving preformationism as the only possible solution to the problem of development \citep{jacob1973_logic}. 

Defenders of preformationism, like Swammerdamm, said that the adult form was already present in the early embryo (or "germ") and that the process of development was just the unfolding of this pre-existent form \citep{amundson2005changing}.
Following this argumentation, it was said that all the germs in the future, present and past existed since the creation, nested one inside of another like Russian dolls, just waiting to be activated \citep{jacob1973_logic}.

Preformationism remained to be the main accepted idea in the 18th century, but some saw its consequences as impossible. Buffon refuted preformationism with a single calculation. He calculated the size that preformed germs of many future subsequent generations should have: for a sixth generation, he calculated, the germ should be smaller than the smallest possible atom \citep{buffon1807buffon}.

\subsection{Haeckel, von Baer and the \textit{Naturphilosophie}}
%The connection between the increase in complexity during development and evolutionary time it has been largely discussed.
In the 19th century, important contributions to embryology were made by advocates of \textit{Naturphilosophie}.
This philosophilcal movement, based in Kant and Goethe's ideas, aimed to classify nature into categories or classes. Among their classification efforts, they classified embryological phenomena and draw analogies between embryos of different taxonomic groups  \citep{Horder2010,Ghiselin2005}.

The first pattern to be recognized, when comparing developmental trajectories of different species, was the Meckel-Serres law. 
This law, named so by E. S. Russell after two of their main proponents: \'{E}tienne Serres and Johann Friedrich Meckel \citep{Russell1916}, proposed that embryos followed a linear succession following the scala naturae.
In this view (influenced by the \textit{Naturphilosophie}), the embryonic development of a higher organism would be a succession of adult forms of lower organisms \citep{Russell1916,amundson2005changing}.

\subsubsection{Karl Ernst von Baer}

K. E. von Baer, an Estonian naturalist considered the father of comparative embryology \citep{Russell1916}, refuted the Meckel-Serres law and formulated his own, known as von Baer's laws \citep{vonBaer1828uber}. 
von Baer's laws stated that general characteristics develop before special characteristics (first law) and that, opposed to the Meckel-Serres law, the embryo of a "higher" animal never resembles the adult of another animal form, but only his embryo (fourth law). 
Importantly, von Baer's views were not evolutionary. The resemblance between developmental trajectories of different species was for him only a reflection of their relationship in the Natural System \citep{amundson2005changing}.
Ironically, in his "Origin of species", Darwin used and reinterpreted von Baer's observations on embryonic stages in different species to support common ancestry and therefore, evolution \citep{darwin1859origin}.

\subsubsection{Ernst Haeckel}
Ernst Haeckel was one of the first who made explicit hypothesis about the connection between development and evolutionary patterns.
He supported Darwinism and, in what is known as Haeckel's "Biogenetic Law", said that development (or ontogeny), is a brief summary of the slow and long phylogeny \citep{haeckel1874menschen}.
In his view, similar to the parallelism view, a "higher" organism would pass through a series of conserved developmental stages that represent ancestral forms (this view is also known as the "recapitulation theory").
However, in contrast with the Meckel-Serres law, he recognized that this recapitulation was almost never complete, due to evolutionary modifications in development. 
He also classified two types of change in development, "heterochrony" and "heterotopy", concepts introduced by him that since then have been crucial in any discussion of the relationship between development and evolution \citep{Horder2013}:
%%%%%%%%%%%%%%%%%%%%%%%%%%%%%%%%%%%%%%%%%%%%%%%%%%%%%%%%%%%%%%%%%%%%%%%%%%%%%%%%%%%%%%%%%%%%%
\begin{flushleft}
\leftskip3em
\rightskip\leftskip
\footnotesize{
\textit{"The falsification of the original course of development is based to a great extent on a gradually occurring displacement of the phenomena, which has been effected slowly over many millennia, by adapting to the changed conditions of embryonic existence. This displacement can affect both their location and time of appearance. Those former we call heterotopy, the latter heterochrony." \citep{haeckel1903anthropogenie}.}}
\end{flushleft}
%%%%%%%%%%%%%%%%%%%%%%%%%%%%%%%%%%%%%%%%%%%%%%%%%%%%%%%%%%%%%%%%%%%%%%%%%%%%%%%%%%%%%%%%%%%%%
Haeckel's views were more complex than usually acknowledged \citep{Richardson2002}.
In fact, he said that it was not that all the mammalian eggs were the same, it was just that with the available tools was impossible to detect the subtle, individual differences, "which are to be found only in the molecular structure" \citep{haeckel1903anthropogenie}.

Now is evident that none of von Baer's or Haeckel's hypothesis can be considered "laws", as they are not universal.
%They only apply to some characters, stages and levels of phylogenetic inclusiveness \citep{Richardson2002}. 
Nevertheless, the works of both Haeckel and von Baer represented the foundations of the comparative embryology field, which is in turn the basis of the modern evolutionary developmental biology (evo-devo).


\subsection{\textit{Entwicklungsmechanik}}

Despite the great advances described above, embryology remained a descriptive science. 
It was not until the end of the 19th century when experimental embryology was born under the name of \textit{Entwicklungsmechanik} (from the german "developmental mechanics"), with the experiments of Roux and Driesch (this is however a simplified version of the origins of \textit{Entwicklungsmechanik}, for a more complete one, see \citealp{Maienschein1991}).

In the 1880's Wilhelm Roux, one of the co-founders of (and coiner of the term) \textit{Entwicklungsmechanik}, performed a simple experiment to test Weismann's theory of inheritance.
%
This theory stated that when a cell divides during development, "chromatin determinants" would be differentially inherited by the daughter cells \citep{Weismann1893}, determining its fate, i.e., if a cell inherits "muscle-determinants" it differentiates into a muscle cell.
This notion of development % (in which each cell self-differentiate)
was called "mosaic development".
Importantly, in Weissman's theory, there is an explicit link between embryology and heredity (or genetics) \citep{gilbert1991conceptual}. In fact, at that time any discussion of development had explicit genetics components, and viceversa \citep{gilbert1991conceptual}.
%Until the beginning of the 20th century, heredity was still considered by many researchers as an aspect of development.
%
%E.G. Conklin, one of the most recognized embryologist of its time 
%"the mechanism of heredity
%can be studied best by the investigation of
%the germ cells and their development"
%
To test the mosaic development hypothesis, Roux killed one blastomere (by puncturing it with a hot needle) in 2-cell frog embryos and observed that, just as Weismann theory predicted, a half embryo was formed \citep{Roux1888}.
%Roux concluded that the embryos followed a mosaic development, composed by self-differentiating parts.

In 1892, in a further attempt to prove mosaic development, Hans Driesch separated the cells of a 2 cell sea urchin blastula with clear expectations of obtaining half sea urchin embryos. 
Instead of this, he was surprised to obtain two small sea urchin embryos \citep{Driesch1892}. One of Driesch's main conclusions was that the fate of a cell was not predetermined after cell division, but it depended on its location in the embryo \citep{driesch1894analytische}. 
Opposite to mosaic development, this type of development has been defined as "regulative development" \citep{Gilbert2014}.
%Driesch turned increasingly from embryology toward philosophy and toward vitalistic views of life which had been since Aristoteles.

The experiments of Roux and Driesch laid the foundations of a new scientific programme whose main purpose was to "research the causes, on which the formation, maintenance and regression of the organic forms are based" \citep{Roux1897}. 
Most importantly, they demonstrated that the problem of development was tractable and that hypotheses could be experimentally tested.

\subsection{Spemann's \textit{organizer}}

In 1921 and 1922, Hans Spemann and Hilde Mangold perfomed what Slack has called "the most famous experiment in all of embryology" \citep{slack2012egg}.
They grafted (transplanted) a part of a gastrula amphibian embryo, the dorsal lip, into different positions of another host embryo. This resulted in the formation of a secondary embryo (that sometimes developed as a siamese twin), partly from the graft and partly from the host embryo \citep{Spemann1924}. They named the dorsal lip region \textit{organizer}.
After its discovery, J. Huxley, G. de Beer, J. Needham and C. H. Waddington had a great influence in spreading the importance of Spemann's findings \citep{Horder2001}. Conrad H. Waddington, a leading embryologist and geneticist mostly know for his 'epigenetic landscape' and 'genetic assimilation' concepts \citep{Slack2002}, wrote:
%%%%%%%%%%%%%%%%%%%%%%%%%%%%%%%%%%%%%%%%%%%%%%%%%%%%%%%%%%%%%%%%%%%%%%%%%%%%%%%%%%%
\begin{flushleft}
\leftskip3em
\rightskip\leftskip
\footnotesize{
\textit{"The special importance of the organization centre is better conveyed by the name Spemann actually chose; it is that part of the embryo with respect to which all the rest is organized. In order to describe the behaviour of any part of a newt gastrula, it is necessary and sufficient to specify its relation to the organization centre. Spemann's name for his discovery may at first sight seem rather grandiloquent, but is really quite reasonable and accurate" \citep{Waddington1962}.} }
\end{flushleft}
%%%%%%%%%%%%%%%%%%%%%%%%%%%%%%%%%%%%%%%%%%%%%%%%%%%%%%%%%%%%%%%%%%%%%%%%%%%%%%%%%%%
However, how the organizer exerted its influence in its surroundings was not known. 
Waddington and many other embryologists around the world tried to characterize the chemical nature of the organizer \citep{Waddington1935,gilbert1991conceptual}.
Despite their efforts, they did not succeed and by the end of the 1930's "the sense of disappointment and disillusionment was manifest" \citep{Horder2010}, which caused the gradual lost of interest in the organizer problem (REF Holftreter in gilbert 1999) 


\subsection{The rise of genetics and its split from embryology}

At the same time Spemann was investigating the organizer, genetics was advancing at a fast pace, establishing its own methods and concepts \citep{gilbert1991conceptual,Horder2001}.
Soon after the rediscovery of Mendel's laws in the 1900's there was an increased acceptance of the chromosomal theory of development. However, many embryologists did not accepted this theory.
Gradually, genetics and embryology began to separate.

A crucial and unexpected contributor to this separation was Thomas Hunt Morgan.
Morgan, who started his career as an embryologist, 
%wrote in 1910: "We have come to look upon the problem of heredity as identical with the problem of development" \citep{Morgan1910}.
first rejected the chromosomal theory (or any particulate theory of development), considering it a modern preformationism view. 
He supported instead an epigenesis view, in which material differences in different eggs (such as chromosomes) "are too remotely connected with the end product of their development for us to think of those differences in terms of special or separate particles except in the purest symbolic fashion" \citep{Morgan1910}.

However, Morgan changed his views on chromosomes and heredity. After the results of his own research on developmental causes on sex determination, and the discovery of many mutations that segregated with the X-chromosome, he was forced to support the view he had been contending against for over a decade \citep{Gilbert1978}.

In his book "Theory of the Gene", Morgan declared the separation between embryology and genetics stating that "the theory of the gene is justified without attempting to explain the nature of the causal processes that connect the gene and the characters" \citep{Morgan1926}.

The new chromosomal theory
%, that stated that the study of genetics was completely different from that of development, 
combined in the 1940's with population genetics and other fields to form the Evolutionary Synthesis. Development, as it was considered irrelevant to the study of heredity, was excluded from the Evolutionary Synthesis \citep{amundson2005changing}. 

Ersnt Mayr, one of the most influential biologists of the 20th century, reinforced in the 1960's the exclusion of development from the Synthesis with his dichotomy of "proximal" and "ultimate" causes \citep{Mayr1961}.
According to Mayr, "proximal" causes like development (or any physiological process) were not of interest for the evolutionary biologist \citep{Mayr1961,Mayr1993}.
%-- calls for reconciliation (waddington urges for integration of genetics and embryology)


\subsection{Developmental genetics}

In the subsequent decades after the Synthesis, there were great advances in molecular biology and genetics. The unravelled DNA structure (REF) and the discovery of the gene regulation of protein synthesis \citep{Jacob1961} lead to the acceptance of the central dogma: DNA must carry the information of Mendelian genes \citep{Crick1958,Crick1970}. 
Genes became the central focus in the study of evolution while development was considered for many to be just a readout of a genetic programme (see \citealp{foxkeller2000geneprogram}).

Taking genes as responsible for the phenotype and assuming, like some evolutionary biologists had, that phylogenetically distant groups of animals developed and had evolved by completely different means \citep{carroll2005endless}, homology between genes was not expected to be found. Mayr wrote:
%%%%%%%%%%%%%%%%%%%%%%%%%%%%%%%%%%%%%%%%%%%%%%%%%%%%%%%%%%%%%%%%%%%%%%%%%%%%%%%%%%%
\begin{flushleft}
\leftskip3em
\rightskip\leftskip
\footnotesize{
\textit{
"Much that has been learned about gene physiology makes it evident that the search for homologous genes is quite futile except in very close relatives. If there is only one efficient solution for a certain functional demand, very different gene complexes will come up with the same solution, no matter how different the pathway by which it is achieved" \citep{Mayr1966}.}}
\end{flushleft}
%%%%%%%%%%%%%%%%%%%%%%%%%%%%%%%%%%%%%%%%%%%%%%%%%%%%%%%%%%%%%%%%%%%%%%%%%%%%%%%%%%%

Mayr's prediction was incorrect. In the 1980's the Hox genes, a family of transcription factors, were shown to be conserved in arthropods and insects \citep{McGinnis1984,Duboule1989}.
Furthermore, Hox genes were shown to be involved in anterior-posterior patterning in many animals.
Thus, not only genes were conserved between different animals, but their developmental role was also conserved. 
The concept of \textit{developmental gene} was born, changing the discussion of development and how the gene was viewed with regard to evolution \citep{gilbert2000developmentalgene}.
The discovery of shared genetic regulatory mechanisms in structures that were not thought to be homologous based on their morphology (like animal eyes) was called "deep homology" \citep{Shubin1997}, a name making clear connections between developmental and evolutionary processes.
%%%%%%%%%%%%%%%%%%%%%%%%%%%%%%%%%%%%%%%%%%%%%%%%%%%%%%%%%%%%%%%%%%%%%%%%%%%%%%%%%%%
\begin{flushleft}
\leftskip3em
\rightskip\leftskip
\footnotesize{
\textit{
"Such homologies provide a profound insight into the evolutionary process.
Studies of deep homology are showing that new structures need not arise from scratch, genetically speaking, but can evolve by deploying regulatory circuits that were first established in early
animals." \citep{Shubin2009} }}
\end{flushleft}
%%%%%%%%%%%%%%%%%%%%%%%%%%%%%%%%%%%%%%%%%%%%%%%%%%%%%%%%%%%%%%%%%%%%%%%%%%%%%%%%%%%

Development was not longer set aside of evolutionary discussions. However, some researchers were convinced that development, not only can be informative of evolutionary processes, but has a causal role in evolutionary change.

\subsection{Evolutionary developmental Biology}

In 1981, 48 researchers from very different scientific backgrounds 
%(e.g., molecular biology, paleontology, developmental genetics, experimental embryology, mathematical biology) 
held a conference in Dahlem (Germany) with one goal: 
\textit{"to examine how changes in the course of development can alter the course of evolution and to examine how evolutionary processes mold development"} \citep{bonner1982evolution}.

\citep{Love2009}.


%The main idea was that all evolutionary change is produced by a change in development (Alberch).

Evolutionary developmental biology (evo-devo) was born.


%\subsection{Why study complexity and adaptation?}
%- Why study it in gene expression?

\subsection{On the statistical approach in Biology}
The statistical approach I have used in here, is nothing but new.

Darwin used a statistical approach to describe the action of natural selection (REF Darwin). For him, given the origination of small variations in natural populations, the occurrence of any advantageous variation in an individual, as slight it could be, would be reflected in a better chance of survival and to procreating their kind (Darwin). With many generations, the differential survival of the variants, would produce a change in the population mean. 
The effects of natural selection are thus only observable at the population level.

A more formal approach came from physics, more precisely from the study of diffusion of gases in the 19th century.

Against the main views of his contemporaries, which considered that all the particles in a gas move at the same speed, J. C. Maxwell proposed that each particle of a gas moved with different velocity and direction, both changing after the particles collision among them (REF Maxwell 1,2).
The velocities in all directions are distributed among the particles according to a certain law. As it was impossible to observe the behaviour of all the particles, their properties could only be described at a statistical level, as the average movement of large numbers of gas particles.

For Boltzmann and Gibbs, which extended the studies on gas diffusion, the study of large numbers was not only important to overcome the problem of not being able to study each individual particles, also because their individual behaviour is not interesting at all (Jacob, logic of life). Knowing the movement and direction of each particle would not give more information than the population as a whole.

After the success of statistical mechanics, its methodology expanded to many other scientific fields.
Laws could be applied to solve previously intractable problems by collecting sufficient information of a great number of cases of the same class and calculating its mean. The aim of the statistical approach is then to "obtain a law which transcends individual cases" (Jacob).

This novel approach changed biology drastically, transforming it into a quantitative science. As Fran\c{c}ois Jacob said, "at the end of the nineteenth century, the study of living beings was no longer a science of order, but one of measurement as well".



%\subsection{Brief outline of the review}

