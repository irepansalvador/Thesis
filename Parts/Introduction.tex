
\subsection{What is Development?}


\subsection{On the roots of Developmental Biology}

In the last decades the scientific community has witnessed the flourishing of developmental biology.
Since the 1980's crucial discoveries (REF Gilbert 1998) not only improved our understanding of the developmental process, but also changed the perspective of the explanatory role of development in biology.
Development roots come from embryology, which until the beginning of the 19th century was mostly a descriptive science.
It was not until the end of the 19th century when experimental embryology was born. In 1894 Wilhelm Roux (REF Roux 1880s), one of the co-founders of experimental embryology (or \textit{"Entwicklungsmechanik"}), performed a simple experiment with which he wanted to test Weismann's theory of inheritance.
This theory stated that when a cell divides during development, "chromatin determinants" would be differentially inherited by the daughter cells. If a cell would inherit "muscle-determinants" it would then differentiate into a muscle cell.
To test this hypothesis, Roux punctured one blastomere of 2-cell frog embryos and obtained, just as Weismann would have predicted, a half embryo of a frog was formed.

Even when this hypothesis was later proved to be wrong, Roux demonstrated that the problem of development was tractable and that hypotheses could be experimentally tested.

The second major approach in development was the genetic approach (examples)

These discoveries prompted discussions on the role of development on the evolutionary theory.

The main idea was that all evolutionary change is produced by a change in development (Alberch).

Evolutionary developmental biology (evo-devo) was born.



\subsection{Why study complexity and adaptation?}

- Why study it in gene expression?

\subsection{On the statistical approach in Biology}
The statistical approach I have used in here, is nothing but new.

Darwin used a statistical approach to describe the action of natural selection (REF Darwin). For him, given the origination of small variations in natural populations, the occurrence of any advantageous variation in an individual, as slight it could be, would be reflected in a better chance of survival and to procreating their kind (Darwin). With many generations, the differential survival of the variants, would produce a change in the population mean. 
The effects of natural selection are thus only observable at the population level.

A more formal approach came from physics, more precisely from the study of diffusion of gases in the 19th century.

Against the main views of his contemporaries, which considered that all the particles in a gas move at the same speed, J. C. Maxwell proposed that each particle of a gas moved with different velocity and direction, both changing after the particles collision among them (REF Maxwell 1,2).
The velocities in all directions are distributed among the particles according to a certain law. As it was impossible to observe the behaviour of all the particles, their properties could only be described at a statistical level, as the average movement of large numbers of gas particles.

For Boltzmann and Gibbs, which extended the studies on gas diffusion, the study of large numbers was not only important to overcome the problem of not being able to study each individual particles, also because their individual behaviour is not interesting at all (Jacob, logic of life). Knowing the movement and direction of each particle would not give more information than the population as a whole.

After the success of statistical mechanics, its methodology expanded to many other scientific fields.
Laws could be applied to solve previously intractable problems by collecting sufficient information of a great number of cases of the same class and calculating its mean. The aim of the statistical approach is then to "obtain a law which transcends individual cases" (Jacob).

This novel approach changed biology drastically, transforming it into a quantitative science. As Fran\c{c}ois Jacob said, "at the end of the nineteenth century, the study of living beings was no longer a science of order, but one of measurement as well".



\subsection{Brief outline of the review}

