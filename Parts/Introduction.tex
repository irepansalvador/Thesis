
This dissertation is based and uses concepts from three main biology fields: 
developmental biology, evolutionary biology and genetics.

%
Nowadays, the union of these scientific fields form multiple research programmes. 
For example, in evolutionary developmental biology (evo-devo), which is the explicit union of the first two fields, at least four major research programmes have been recognized (Muller, 2006).

However, these fields have not always gotten along well. 
Few decades ago, there was a separation between the evolutionary biology and developmental biology, even when embryology was considered very important for the study of evolution in the 19th century.

As a matter of introduction to this dissertation, in the following section I will first aboard the question defining development and then I will make a brief introduction of the scientific (and philosophical) origins of developmental biology, with special attention to its relations with evolutionary biology and genetics. 

This is not intended to be a comprehensive review, for a more complete view see (admudson, gilbert..)

%Thesis main subjects:
%increase of complexity in development with an evolutionary perspective
%adaptation in the embryo


\subsection{What is Development?}

There is no unique or straightforward answer to this apparently simple question.
Traditionally, the problem of development has been studied by embryologists. However, embryonic development does not necessarily equate to development,
%theDevelopment is sometimes equate to embryonic development, probably due to the fact that developmental biology origins come from embryology.
%Equating embryonic development to development is problematic 
specially when considering organisms with complex life cycles. For example, holometabolous insects, in addition to embryonic development, undergo a complete metamorphosis (from pupa to adult), a process that could be considered a second embryonic development.

Currently, the most common definition of development refers to the set of processes through which an egg is transformed into an adult (minelli 2011).
In 1880 Haeckel defined development in similar terms: "individual development, or the ontogenesis of every single organism, from the egg to the complete form is nothing but a growth attended by a series of diverging and progressive changes" (Haeckel book).

This egg-to-adult view has been criticized by some authors to be an "adultocentric" view of development, and suggest instead to consider within the boundaries of development the whole life cycle of an organism (Gilbert 2011, Minelli 2011).
In the view of Julian S. Huxley and Gavin R. de Beer, "development is not merely an affair of early stages; it continues, though usually at a diminishing rate, throughout life" (Huxley and de Beer).

%However, that this concept is not applicable to some organisms (Minelli in book). Some animals or plants instead of having an egg or seed stage as means of reproduction have buds or other vegetative parts.

But even after considering the whole life cycle complications appear in cases in which the common notion of development of an individual organism would not apply, as in polyembryonic development or colonial organisms (Minelli).
There have been recent attempts to construct a broader concept of development (ref in book Moscek, Griesemeser, Pradeu). Moczek, for example, defines development as "the sum of all processes and interacting components that are required to allow organismal form and function, on all levels of biological organization, to come into being" (Moczek in book).

The main challenge on adopting a new concept of development which is more inclusive, is to maintain its intuitiveness and applicability in scientific research.

In here I will use the "common view" of development (Minelli book ref2), that considers the egg and the adult as the start and end of individual development respectively, even when for practical reasons, some analysis of this dissertation (article 1 and 3) include only embryonic development. 


\subsection{On the history of Developmental Biology}

In the last decades the scientific community has witnessed the flourishing of developmental biology.
Since the 1980's crucial discoveries (REF Gilbert 1998) not only improved our understanding of the developmental process, but also changed the perspective of the explanatory role of development in biology.

However, the roots of developmental biology, which go back many centuries come from embryology and anatomy.
%The roots of the modern development biology come from embryology and anatomy.
%In the next paragraphs I make a brief introduction of the scientific (and philosophical) origins of developmental biology.
For simplicity, I have divided the history of developmental biology in four periods: from Aristotle to the 18th century, the 18th century preformationism-epigenesis debate, the study of "developmental mechanics" in the 19th century, the molecular genetics and the emergence of evolutionary developmental biology (evo-devo).
The summary presented here grasps only the surface of this issue and is not intended to be comprehensive, for further and deeper lecture, see (gilbert, Amudson, Hall?, Jacob, etc)


\subsubsection{Aristotle}
%which until the beginning of the 19th century was mostly a descriptive science.
Before the 19th century, the major single contributor in the study of embryology was Aristotle. 
Some of his most important contributions to embryology are:

%%%%%%%%%%%%%%%%%%%%%%%%%%%%%%%%%%%%%%%%%%%%%%%%%%%%%%%%%%%%%%%%%%%%%%%%%%%%%%%%%%%%%%%
\begin{enumerate}
\item He organized and classified animals accordingly to their embryonic development (Aristotle ) after careful observation of the development in many species. Because of this, he has been considered the first comparative embryologist (J. Needham, A History of Embryology).
\item For him, developmental process was driven by "internal causes" and required a "soul" that guides the process of development.
\item He clearly defined two opposite theories of development, preformationism and epigenesis, from which he supported the latter.
\end{enumerate}
%%%%%%%%%%%%%%%%%%%%%%%%%%%%%%%%%%%%%%%%%%%%%%%%%%%%%%%%%%%%%%%%%%%%%%%%%%%%%%%%%%%%%%%

After Aristotle the preformationism-epigenesis debate would last centuries attracting many of the most important philosophers and naturalists.

\subsubsection{The preformationism-epigenesis debate}

Until the 18th century, supporters of epigenesis (like Wolffs and ??) saw development as starting from a formless embryo, with its form arising during development following a "vital" force (amudson).

During the 18th century, many rejected any vital force to explain development, leaving preformationism as the only possible solution to the problem of development (Jacob). Defenders of preformationism (like Swammerdamm and ??) said that the form of the adult was already present in the early embryo (or "germ") and that the process of development was just the unfolding of this pre-existent form (amundson). 
Following this argumentation, it was said that all the germs in the future, present and past existed since the creation (nested one inside of another like Russian dolls) which were just waiting to be activated (jacob).

Even when preformationism remained to be as the main accepted idea in the 18th century, some scientists saw its consequences as impossible. Buffon made a calculation of the size of the preformed germs of many subsequent generations. He calculated that for a sixth generation, its germ would be smaller than the smallest possible atom (Buffon,Natural History, vol. III,).


\subsubsection{\textit{Entwicklungsmechanik}}

Despite the persistent preformationism-epigenesis debate, until the beginning of the 19th century, embryology was mostly a descriptive science. It was not until the end of the 19th century when experimental embryology was born under the name of \textit{Entwicklungsmechanik} (from the german "developmental mechanics"), with the famous experiments of Roux and Driesch (this is however a simplified version of the origins of Entwicklungsmechanik, for a more complete one see Maienschein in Gilbert).

In 1888? Wilhelm Roux (REF Roux 1880s), one of the co-founders of experimental embryology and coiner of the term \textit{Entwicklungsmechanik}, performed a simple experiment with which he wanted to test Weismann's theory of inheritance.

This theory stated that when a cell divides during development, "chromatin determinants" would be differentially inherited by the daughter cells, without any external influences (Weismann 1893), therefore, if a cell inherit "muscle-determinants" it will then differentiate into a muscle cell. 
This notion of development (in which each cell self-differentiate) was called "mosaic development".

Importantly, in this hypothesis there is an explicit link between embryology and heredity. 
Until the beginning of the 20th century, heredity was still considered by many researchers as an aspect of development.

E.G. Conklin, one of the most recognized embryologist of its time 
"the mechanism of heredity
can be studied best by the investigation of
the germ cells and their development"

To test the mosaic development hypothesis, Roux killed one blastomere (by puncturing it with a hot needle) in 2-cell frog embryos and observed that, just as Weismann theory predicted, a half embryo was formed.
%Roux concluded that the embryos followed a mosaic development, composed by self-differentiating parts.
In 1892, in a further attempt to prove mosaic development, Driesch separated the cells of a 2 cell sea urchin blastula with clear expectations of obtaining half sea urchin embryos. 
Instead of this, he was surprised to obtain two small sea urchin embryos (REF). One of Driesch main conclusions were that the fate of a cell was not predetermined after cell division, but it depended on its location in the embryo (Gilbert?, Driesch 1894). 
Opposite to mosaic development, this type of development has been defined as "regulative development" (Gilbert).
%Driesch turned increasingly from embryology toward philosophy and toward vitalistic views of life which had been since Aristoteles.
Even when Roux's hypothesis was proved wrong by Driesch just a few years after being proposed, it demonstrated that the problem of development was tractable and that hypotheses could be experimentally tested.

\subsubsection{The molecular approach}



The second major approach in development was the genetic approach (examples)

These discoveries prompted discussions on the role of development on the evolutionary theory.

The main idea was that all evolutionary change is produced by a change in development (Alberch).

Evolutionary developmental biology (evo-devo) was born.



\subsection{Why study complexity and adaptation?}

- Why study it in gene expression?

\subsection{On the statistical approach in Biology}
The statistical approach I have used in here, is nothing but new.

Darwin used a statistical approach to describe the action of natural selection (REF Darwin). For him, given the origination of small variations in natural populations, the occurrence of any advantageous variation in an individual, as slight it could be, would be reflected in a better chance of survival and to procreating their kind (Darwin). With many generations, the differential survival of the variants, would produce a change in the population mean. 
The effects of natural selection are thus only observable at the population level.

A more formal approach came from physics, more precisely from the study of diffusion of gases in the 19th century.

Against the main views of his contemporaries, which considered that all the particles in a gas move at the same speed, J. C. Maxwell proposed that each particle of a gas moved with different velocity and direction, both changing after the particles collision among them (REF Maxwell 1,2).
The velocities in all directions are distributed among the particles according to a certain law. As it was impossible to observe the behaviour of all the particles, their properties could only be described at a statistical level, as the average movement of large numbers of gas particles.

For Boltzmann and Gibbs, which extended the studies on gas diffusion, the study of large numbers was not only important to overcome the problem of not being able to study each individual particles, also because their individual behaviour is not interesting at all (Jacob, logic of life). Knowing the movement and direction of each particle would not give more information than the population as a whole.

After the success of statistical mechanics, its methodology expanded to many other scientific fields.
Laws could be applied to solve previously intractable problems by collecting sufficient information of a great number of cases of the same class and calculating its mean. The aim of the statistical approach is then to "obtain a law which transcends individual cases" (Jacob).

This novel approach changed biology drastically, transforming it into a quantitative science. As Fran\c{c}ois Jacob said, "at the end of the nineteenth century, the study of living beings was no longer a science of order, but one of measurement as well".



%\subsection{Brief outline of the review}

